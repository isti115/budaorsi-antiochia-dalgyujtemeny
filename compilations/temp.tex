\documentclass[a5paper,twoside]{article}
\usepackage[utf8]{inputenc}
\usepackage{lmodern}
\usepackage[T1]{fontenc}
\usepackage[hungarian]{babel}

\usepackage{graphicx}
\graphicspath{ {images/} }

\usepackage{multicol}
\usepackage[bookmarks]{hyperref}

\usepackage{xifthen}
\newif\ifchordtmp

\usepackage[chorded]{songs}
% \usepackage[lyric]{songs}
% \usepackage[slides]{songs}

% \title{RáBalics nászmise\\\textit{(2021. október 9.)}}
\title{Juniális\\\textit{(2022. június 12.)}}
% \author{Donkó István}
\date{}

\setlength{\topmargin}{-1.5cm}

\setlength{\headheight}{0cm}
\setlength{\headsep}{0cm}
\setlength{\footskip}{0.85cm}

\setlength{\oddsidemargin}{-1.8cm}
\setlength{\evensidemargin}{-1.8cm}

\setlength{\textwidth}{13.25cm}
\setlength{\textheight}{18.25cm}
% \setlength{\textheight}{19cm}

\setlength{\songnumwidth}{1.02cm}
\setlength{\versenumwidth}{0.5cm}
% \setlength{\cbarwidth}{0.1cm}

\catcode`_=12 % use underscore as regular character
\renewcommand{\_}[1]{\underline{#1}} % shorter command for adding melismas
\newcommand{\refrain}[2][]{
  \beginchorus
    \ifchorded \chordtmptrue \chordsoff \fi
      \textit{\textbf{Refrén}}
      \ifthenelse{\isempty{#1}}{}{\rep{#1}}
    \ifchordtmp \chordtmpfalse \chordson \fi
    #2
    % this empty line below is important!

  \endchorus
} % command for chorus with optional repetition and additional text / chords

% \newcommand{\var}[1]{\mbar{\footnotesize $^#1$}{}}
\newcommand{\var}[1]{\mbar{\footnotesize #1}{}}
\songcolumns{1}
\notenames{A}{H}{C}{D}{E}{F}{G}

\begin{document}
  % \pagenumbering{gobble}
  \pagenumbering{arabic}

  % \begin{titlepage}
  %   \pagenumbering{gobble}
  %   \setlength{\oddsidemargin}{-1.625cm}
  %
  %   \vspace*{4cm}
  %   % \vspace{-1.5cm}
  %   {\let\newpage\relax\maketitle}
  %   % \vspace{-1.5cm}
  % \end{titlepage}

  \versesep=12pt plus 3pt minus 5pt

  \iflyric
    \baselineadj=2pt plus 1pt minus 1pt
  \fi

  \begin{center}
    \vspace*{2cm}
    {\huge Juniális} \\
    \vspace*{0.2cm}
    {\Large \textit{(2022. június 12.)}}
    \vspace{3cm}
  \end{center}
  \vspace{-0.25cm}

  % \begin{center}
  %   % \vspace*{1.5cm}
  %   {\huge RáBalics nászmise} \\
  %   \vspace*{0.1cm}
  %   {\Large \textit{(2021. október 9.)}}
  %   % \vspace{2cm}
  % \end{center}
  \vspace{-0.25cm}

  % \includeonlysongs{M11,101,56,M11,101,56}
  % \includeonlysongs{80,M10,55,26,M30,M40,85,40,15}
  % \includeonlysongs{46,M10,M20,84,20,56,M30,M40,87,48,97,73,40,92}
  % \includeonlysongs{6,M10,M20,888,T101,999,26,M30,M40,87,88,85,118}
  % \includeonlysongs{M11,87,125,56,M31,M42,91,59,T2}
  % \includeonlysongs{76,M10,M20,T12,???,24,M31,M42,68,87,40} % 2022. feb.
  % \includeonlysongs{56,M10,16,T8,24,M30,M41,25,48,4}
  \includeonlysongs{14,M10,M21,T102,T101,56,M30,M41,91,77,85,73,102}
  % \includeonlysongs{46,M10,M20,M21,T3,7,85,M31,M42,89,71,4}
  % \includeonlysongs{T101}
  % \includeonlysongs{999}
  % \includeonlysongs{999,M41}
  % \includeonlysongs{54,90,36,80,91,77,97,40,84,11,6}

  % \includeonlysongs{45,50,74,77,63,65,88,87,91,96,T1,10,97}

  \begin{songs}{}
    \beginsong{Add tovább}
[
  index={Nagy tűz, hatalmas láng}
]

  \ifchorded
    \beginverse*
      {\nolyrics Előjáték: \[D] \[F#m] \[G] \[A] \[A7]}
    \endverse
  \fi

  \beginverse\memorize
    Nagy \[D]tűz, hatalmas \[F#m]láng, csak egy \[G]szikra az, amitől \[A]kigyullad. \echo{amitől \[A7]kigyullad}
    \[D]Melegét szórja \[F#m]rád, csak a \[G]kezed kell, hogy oda\[A]ny\_{ú}jtsad. \echo{hogy oda\[A7]ny\_{ú}jtsad}
  \endverse

  \beginchorus
    \[G]Épp így az Isten \[D]szeretetét, \[Em] ha egyszer el\[A]fogad\[D]tad,
    Már \[G]nem kí\[D]vánsz \[G]semmi \[D]mást, csak \[G]azt, hogy \[A]add to\[D]vább.
  \endchorus

  \beginverse
    Az ^első napsu^gár, mi a ^dermedt tavaszi ^földre száll. \echo{tavaszi ^földre száll}
    ^Melegét szórja ^rád, és a ^fák rügyeiket ^kibontják. \echo{^kibontják}
  \endverse

  \beginverse*
    Refrén
  \endverse

  \beginverse
    ^Szívemben boldog^ság, mert az ^Úr, az Isten az ^én Atyám. \echo{igen, az ^én Atyám}
    ^Lelkemben ég a ^vágy, h\_{o}gy ^megtegyem az Ő ^akaratát. \echo{az Ő ^akaratát}
  \endverse

  \beginverse*
    /: A \[G]hegy csúcsáról \[D]kiáltok: Uram! \[Em] Tárd fel a \[A]vilá\[D]gom!
    \[G]Hogy to\[D]vább \[G]adjam \[D]én a \[G]szeretet \[A]Iste\[D]nét. :/ \rep{2}

    % /: A \[G]hegy csúcsáról \[D]kiáltok: Uram! \[Em]Tárd fel a \[A]vilá\[D]gom!
    % Hogy \[G]tovább \[Em]adjam \[A]én a \[G]szeretet \[A]Iste\[D]nét. :/ \rep{2}
  \endverse

\endsong

\beginsong{Adjunk hálát}[]

  \versesep=12pt plus 3pt minus 8pt

  \ifchorded
    \beginverse*
      {\nolyrics Előjáték: \[G] \[C] \[G]}
    \endverse
  \fi

  \beginverse\memorize
    Adjunk \[C]há\[G]lát az \[C]Úr\[G]nak,
    Aki \[Em]meghalt \[Am]és fel\[D]támadt
    Ő a \[C]mennybe \[G]ment és a \[C]trónon \[G]ül,
    Onnan \[Em]várjuk, \[Am]míg el\[D]jő.
  \endverse

  \beginchorus
    Allelu\[Hm]ja, az Úr az \[Em]Isten!
    Allelu\[Am]ja, a megvál\[D]tó!
    Allelu\[Hm]ja, az Úr az \[Em]Isten!
    Zengj alle\[Am]luját! \[D7]Á\[G]men! \[C] \[G]
  \endchorus

  \beginverse
    Az Ő ^tes^te és ^vé^re
    Táplál ^minket ^itt a ^Földön,
    Gyere ^és ö^rülj, velünk ^ünne^pelj,
    Tiszta ^szívvel ^áldjuk ^Őt!
  \endverse

  \beginverse*
    Refrén
  \endverse

  \beginverse
    Aki ^benne ^hisz, sose ^szomja^zik,
    Ő az ^élő ^víz fo^rrása.
    Szent^lélek, ^jöjj és ^szállj le ^ránk,
    Tölts be ^minket ^szeretet! ^
  \endverse

  \beginverse*
    Refrén
  \endverse

\endsong

\beginsong{Áldjátok az Urat}[]

  \beginverse*
    /: \[Em]Áldjátok az Ur\[Am]at, áldjátok szent ne\[Em]vét,
    Kik \[Em]házában álltok \[D]napról n\underline{a}pra, \[Hm]dicsérjétek szüntelen \[Em]Őt! :/ \rep{2}
  \endverse

  \beginverse*
    /: \[Em]Tárjátok kezei\[Am]tek az élő Isten fe\[Em]lé!
    \[Am]Áldjon meg téged \[Em]Sionból az Úr, \[D]ragyogtassa \[Hm]arcát re\[Em]ád! :/ \rep{2}
  \endverse

\endsong

\beginsong{Antióch dal}[]

  \beginverse
    \[G]Gyere, gyere, gyere és tapsoljál velünk! \echo{pop subap, pop pop subap}
    \[C]Gyere, gyere, gyere és tapsoljál velü\[G]nk! \echo{pop subap, pop pop subap}
    \[D]Gyere, gyere, gyere és énekelj, hogy \[C]jól szóljon \[C\⁷] a dalunk:
  \endverse

  \beginchorus
    \[G] A-N-\[Em]T, A-N-T,
    \[G] A-N-\[Em]T-I-O-H,
    \[G] A-N-\[Em]T-I-\[C]O-\[D]H-I-\[G]A
  \endchorus

  \beginverse
    ^Gyere, gyere, gyere és rázd a jobb kezed! \echo{pop subap, pop pop subap}
    ^Gyere, gyere, gyere és rázd a bal ke^zed! \echo{pop subap, pop pop subap}
    ^Gyere, gyere, gyere és énekelj, hogy ^jól szóljon ^ a dalunk:
  \endverse

  \refrain

  \beginverse*
    \[C] Amikor ti hívtatok, \[G]úgy éreztem, kő vagyok,
    \[C]De itt rátok találtam, mit \[D]úgy, úgy, de \[D#]úgy vár\[D]tam!
  \endverse

  \beginverse
    ^Gyere, gyere, gyere és táncoljál velünk! \echo{pop subap, pop pop subap}
    ^Gyere, gyere, gyere és táncoljál velü^nk! \echo{pop subap, pop pop subap}
    ^Gyere, gyere, gyere és énekelj, hogy ^jól szóljon ^ a dalunk:
  \endverse

  \refrain

\endsong

\beginsong{Bízom Benned}[]

  \ifchorded
    \beginverse*
      {\nolyrics Előjáték: \[G] \[C9] \[Em7] \[D]}
    \endverse
  \fi

  \beginchorus
    \[G]Bízom \[C9]Benned, \[Em7]igazságos \[D]Úr vagy,
    \[G]Bízom \[C9]Benned, \[Em7]irgalmas, \[D]jó!
    \[G]Bízom \[C9]Benned \[Em7]minden\[D]nap, örö\[G]kké! \[C9] \[Em7] \[D]
  \endchorus

  \beginverse\memorize
    \[G] Olyan messziről jöttem \[Am7]én,
    És Te \[F]megvár\[C]tál, sokat \[G] vártál \[D]rám,
    \[G] Úgy hívlak a magány éjje\[Am7]lén,
    Mikor \[F]elvész a \[C]szó, mikor csak \[G]átölelni \[D]jó!
    \[Am]Tudom, hogy \[Em]távol voltam, \[G]fájt Ne\[D]ked.
    \[Am]Tudom, hogy \[Em]nélküled nem \[C]élhe\[D]tek.
  \endverse

  \beginverse*
    Refrén
  \endverse

  \beginverse
    ^ Nézd, a szívem már Ti^éd!
    Olyan ^boldog^ság, mikor ^ Téged ^lát!
    ^ Hallod hangom lágy ne^szét,
    Kérlek, ^szólj hoz^zám, mikor csak ^Téged hív a ^szám!
    ^Életem ^nem féltem, hisz ^jó Ve^led.
    ^Tudom, hogy ^nélküled nem ^élhe^tek!
  \endverse

  \beginverse*
    Refrén \rep{2}
  \endverse

\endsong

\beginsong{Csak Jézusnak szolgálok}[]

  \beginchorus
    Csak \[G]Jézus\[D]nak \[Em]szolgá\[C]lok, az \[G]Ő or\[C]szágát \[G]vá\[D]rom,
    Mert \[G]eljön \[D]Ő, az \[C]Üdvözí\[D#]tő, ó \[G]Jézus, \[D]feltámadt Ki\[G]rály! \[C] \[G]
  \endchorus

  \beginverse\memorize
    \[Em]Áldjad, lelkem \[Hm]Őt, \[Em]dicsérd Terem\[Hm]tőd, \[G]most és \[C]mindörökké, \[G]á\[D]men!
    \[Em]Áldjad, minden \[Hm]nép, \[C]dicsérd szent Ne\[G]vét, \[G]most és \[C]mindörökké, \[D]á\[G]men!
  \endverse

  \beginverse
    ^Hálát adja^tok, ^áldást mondja^tok, ^most és ^mindörökké, ^á^men!
    ^Egyházad di^csér ^nagy jóságo^dért, ^most és ^mindörökké, ^á^men!
  \endverse

  \beginverse
    ^Erős vár az ^Úr, ^menedéket ^nyújt, ^most és ^mindörökké, ^á^men!
    ^Áldott irga^lom, ^véred olta^lom, ^most és ^mindörökké, ^á^men!
  \endverse

  \beginverse
    ^Hatalmas az ^Úr, ^jóságos az ^Úr, ^most és ^mindörökké, ^á^men!
    ^Szívem áldja ^Őt, ^ujjongjon a ^Föld, ^most és ^mindörökké, ^á^men!
  \endverse

  \beginverse
    ^Szentek, angya^lok, ^szeráfkóru^sok, ^most és ^mindörökké, ^á^men!
    ^Néked zenge^nek, ^így ünnepel^nek, ^most és ^mindörökké, ^á^men!
  \endverse

  \beginverse
    ^Áldott légy ^Atya, ^s egyszülött ^Fia, ^most és ^mindörökké, ^á^men!
    ^Éltető Lé^lek, ^Téged dicsér^lek, ^most és ^mindörökké, ^á^men!
  \endverse

\endsong

\beginsong{Ébredj ember}
[
  li={https://youtu.be/-DGzHCfmv5k alapján.}
]

  \versesep=12pt plus 3pt minus 8pt

  \ifchorded
    \beginverse*
      {\nolyrics Előjáték: \[C] \[F] \[C] \[C] \[F] \[G] \[Am] \[F] \[C] \[Em] \[Am] \[Dm] \[G] \[C]}
    \endverse
  \fi

  \beginverse
    Ébredj \[C]ember \[G] mély álmod\[Am]ból, \[F] Jézus \[C]megment \[Dm] rabságod\[G]ból!
    Hogyha \[Am]hívod, \[Em] eljön \[F]Ő, egész \[C]bensőm \[Dm] imádjad \[G]Őt!
  \endverse

  \beginchorus
    Alle\[C]luja, alle\[F]lu\[C]ja!
    Alle\[C]luja, alle\[F]lu\[G]ja!
    Alle\[Am]luja, alle\[F]luja, al\[C]le\[Em]lu\[Am]ja!
    Alle\[Dm]luja, alle\[G]lu\[C]ja!
  \endchorus

  \beginverse*
    {\nolyrics \echo{Két félhang emelkedés.} \[A7]}
  \endverse

%   \textnote{két félhang emelkedés}

  \beginchorus
%     {\nolyrics (két félhang emelkedés) \[A7]}
    Alle\[D]luja, alle\[G]lu\[D]ja!
    Alle\[F#m]luja, alle\[G]lu\[A]ja!
    Alle\[Hm]luja, alle\[G]luja, al\[D]le\[F#]lu\[Hm]ja!
    Alle\[Em]luja, alle\[A]lu\[D]ja!
  \endchorus

  \beginverse\memorize
    Áldjad \[D]lelkem \[A] az Úr ne\[Hm]vét, \[G] mert meg\[D]tartja \[Em] ígére\[A]tét!
    Mennyi \[Hm]jót tett \[F#m] az Úr ve\[G]lem, Reá \[D]bízom \[Em] az éle\[A]tem.
  \endverse

%   \beginverse
%     Föltámadt Jézus, győztes király, elhozta nékünk az országát.
%     Jöjj Szentlélek, szállj közénk, hozz a mennyből szép, tiszta fényt!
%   \endverse

  \beginverse*
    Refrén
  \endverse

  \beginverse
    És az ^élet, ^íme, megjele^nt, ^ köztünk ^él Ő ^ Szentlelké^ben.
    Hív Ő ^téged, ^ hogy élj ve^le, add át ^néki ^ a szíve^det!
  \endverse

  \beginverse*
    Refrén
  \endverse

\endsong

\beginsong{A fény, ami bennem ég - !!}[]

  \beginverse\memorize
    A \[G]fény, ami bennem ég nem fog már kihunyni,
    A \[C]fényt, ami bennem ég, \[G]égni hagyom már. \echo{Hej babám!}
    A fény, ami bennem ég \[H7]sugározzon \[Em]szét,
    Legyen \[G]fény, legyen \[D]fény, legyen \[G]fény! \[C] \[G] \[D]
  \endverse

  \beginverse
    A fény, ami Krisztusé…
  \endverse

  \beginverse
    A ^világ minden sarkában fénynek kell kigyúlni,
    A ^világ minden sarkában ^éghetne a láng. \echo{Hej babám!}
    A világ minden sarkába a ^fényt elvihet^ném,
    Legyen ^fény, legyen ^fény, legyen ^fény! ^ ^ ^
  \endverse

  \beginverse
    A Szent Imre minden sarkában…
  \endverse

  \beginverse
    Budaörs minden sarkában…
  \endverse

  \beginverse
    A Szent Ferenc minden sarkában…
  \endverse

  \beginverse
    Szentendre minden sarkában…
  \endverse

  \beginverse
    Fehérvár minden sarában…
  \endverse

  \beginverse
    Kanizsa minden sarkában…
  \endverse

  \beginverse
    Szeged minden sarkában…
  \endverse

  \beginverse
    Kaposvár minden sarkában…
  \endverse

  \beginverse
    Pécs minden sarkában…
  \endverse

\endsong

\beginsong{Föltámadt! Alleluja!}
[
  by={Sillye Jenő},
  index={Álltok szótlanul}
]

  % \ifchorded
  %   \beginverse*
  %     {\nolyrics Előjáték: \[E] \[A] \[A7]}
  %   \endverse
  % \fi

  \beginverse\memorize
    \[D]Álltok \[A7]sz{\_{ó}}tla\[D]nul, a \[G]dárda \[A]földre \[D]hull, \[Hm]
    Itt \[G]más e\[A]rő az {\[D Hm]{\_{Ú}}}r, itt \[Em]más e\[A]r{\_{ő}} az \[D]Úr!
  \endverse

  \beginverse
    Az ^éj az ^hosszú ^volt, de ^eltűnt ^már a ^Hold, ^
    A ^hajnal {^felsik}{^{\_{o}}}lt, a ^hajnal ^f{\_{e}}lsi^kolt: \[D7]
  \endverse

  \beginchorus
    \[G]Föltá\[A]madt! Allel{\[D Hm]{\_{u}}}ja! \[Em]Föltámadt! \[A7]Allel\_{u}\[D]ja! \[Hm]
    \[G]Föltá\[A]madt! Allel{\[D Hm]{\_{u}}}ja! \[Em]Föltámadt! \[A7]Allel\_{u}\[D]ja!
  \endchorus

  \beginverse
    ^Elég volt ^m{\_{á}}r, e^lég, ^fénnyel ^hinti az ^ég ^
    Az ^új nap {^reggel}{^{\_{é}}}t, az ^új nap ^r{\_{e}}gge^lét.
  \endverse

  \beginverse
    ^Virágban ^{\_{á}}ll a ^rét, a ^virágos ^úton ^át ^
    ^Valaki ^jön fel{^{\_{é}}}d, ^Valaki ^j{\_{ö}}n fe^léd. \[D7]
  \endverse

  \beginverse*
    Refrén
  \endverse

  \beginverse
    ^Virágban ^{\_{á}}ll a ^rét, a ^virágos ^úton ^át ^
    ^Jézus ^jön fel{^{\_{é}}}d, ^Jézus ^j{\_{ö}}n fe^léd. \[D7]
  \endverse

  \beginverse*
    Refrén \rep{2}
  \endverse

\endsong

\beginsong{Ha jön az Úr}[]

  \ifchorded
    \beginverse*
      {\nolyrics Előjáték: \[E] \[E7] \[A] \[Am] \[E] \[H7] \[E] \[A] \[E]}
    \endverse
  \fi

  \beginverse\memorize
    Ha jön az \[E]Úr, ha visszatér, ha jön az Úr, ha vissza\[H7]tér,
    Hívj, A\[E]tyám, a \[E7]szentek \[A]közé, \[Am] ha jön az \[E]Úr, ha \[H7]vissza\[E]tér! \[A] \[E!]
  \endverse

  \beginverse
    Ha minden ^szent életre kél, ha minden szent életre ^kél,
    Hívj, A^tyám, a ^szentek ^közé, ^ ha minden ^szent é^letre ^kél! ^ ^
  \endverse

  \beginverse
    Ha zengik ^mind, alleluja, ha zengik mind, allelu^ja,
    Hívj, A^tyám, a ^szentek ^közé, ^ ha zengik ^mind al^lelu^ja! ^ ^
  \endverse

  \beginverse
    Ha ég és ^Föld új arcot ölt, ha ég és Föld új arcot ^ölt,
    Hívj, A^tyám, a ^szentek ^közé, ^ ha ég és ^Föld új ^arcot ^ölt! ^ ^
  \endverse

  \beginverse
    A mennyben ^fenn, oly' nagy a csend, elvétve egy-két hárfa ^zeng,
    De nem így ^lesz, ó, ^alle^luja, ^ ha egyszer ^én is ^ott le^szek! ^ ^
  \endverse

  \beginverse
    Nem sakko^zik, nem golfozik, egy rendes angyal nem i^szik,
    Így éne^kel, ó ^alle^luja, ^ miközben ^rock n' ^rollo^zik! ^ ^
  \endverse

\endsong

\beginsong{Hús-vér templom}
[
  by={Pintér Béla}
]

  \ifchorded
    \beginverse*
      {\nolyrics Előjáték: /: \[Dm] \[A#] \[C] \[F] :/ \rep{2}}
      % TODO: A# helyett Gm?
    \endverse
  \fi

  \beginchorus
    /: Nekem \[Dm]nincs más \[Gm]Rajtad \[C]kívül, \[F]Jézus,
    \[Dm]Éle\[Gm]tem for\[C]rása \[F]vagy.
    \[Dm]Kézzel, \[Gm7]lábbal, \[A]szívvel, \[Dm]szájjal
    \[A#]Dicsérlek, Ura\[A4]m. \[A] :/ \rep{2}
  \endchorus

  \beginverse
    Ez a \[Dm]hús-vér \[Gm]templom \[A]Érted \[Dm]épült,
    Neked \[A#]ég a \[Gm7]tűz bent, \[A]oltárai\[Dm]nál.
    Ez a hús-vér \[Gm]templom \[A]Téged \[Dm]dicsér,
    Szívem \[A#]minden \[Gm7]húrja \[A]Rólad muzsi\[Dm]kál.
  \endverse

  \beginverse*
    Refrén
  \endverse

\endsong

\beginsong{Ím, itt jön Ő}[]

  \ifchorded
    \versesep=12pt plus 3pt minus 6pt
  \fi

  \ifchorded
    \beginverse*
      {\nolyrics Előjáték: \[Em] \[Hm] \[Em] \[C] \[Hm] \[Em]}
    \endverse
  \fi

  \beginchorus
    /: Ím, \[Em]itt jön Ő, átkel \[Hm]a hegyen,
    A \[Em]dombon át sza\[C]lad felém a \[Hm]kedve\[Em]sem! :/ \rep{2}
  \endchorus

  \beginverse\memorize
    Hát \[G]jöjj, Uram, lásd, Ti\[D]éd a szívem,
    A \[Em]kerted Téged vár, a szőlőd \[H]is terem.
    A \[G]szív dalol, zengi \[D]énekét:
    A \[Em]lelkem ég, ó \[C]Istenem, sze\[Hm]relme\[Em]dért.
  \endverse

  \refrain

  \beginverse
    Úgy ^vágyom, Uram, ünne^pelni Veled,
    Meg^részegít a bor s a méz, mit ^adsz nekem.
    Ha ^elfogadod minden ^kincsemet,
    Én ^átadom az ^életem, Ti^éd le^gyen!
  \endverse

  \refrain

  \beginverse
    Az ^Úr elé együtt ^indulunk,
    Ő ^mindig újra vár miránk, ha ^elbukunk.
    A ^nász idején ünne^peljetek,
    Hisz ^itt a vőle^gény, ki minket ^úgy sze^ret!
  \endverse

  \refrain{\echo{A \[Hm]kedve\[Em]sem, a \[Hm]ked\[Hm\⁷]ve\[Em]sem.}}

\endsong

\beginsong{Indulj a világba -- Előjáték jelölés??? - kell "És" az "az Úr elé?"}[]

  \ifchorded
    \beginverse*
      {\nolyrics Előjáték: \[D] \[D-=Db] \[D] \[D-=Db]}
    \endverse
  \fi

  \beginverse\memorize
    \[D]Én vagyok, ki a közösséget építi,
    És \[A]én vagyok, ki a közösséget építi,
    És \[D]én vagyok, ki a közösséget építi,
    És \[A]tovább szeretném ad\[D]ni!
  \endverse

  \beginchorus
    \[G]Indulj a világba s \[D]menjél bármerre:
    \[A]magas havasokba, s a kék \[D]tengerre,
    \[G]Indulj a világba, az \[D]Úr van teveled,
    \[A]Indulj hát és építsd a kö\[D]zös\[A]sé\[D]get!
  \endchorus

  \beginverse
    ^Te vagy az, ki a közösséget építi,
    És ^te vagy az, ki a közösséget építi,
    És ^te vagy az, ki a közösséget építi,
    És ^tovább szeretné ad^ni!
  \endverse

  \beginverse
    ^Mi vagyunk, kik a közösséget építjük,
    És ^mi vagyunk, kik a közösséget építjük,
    És ^mi vagyunk, kik a közösséget építjük,
    És ^tovább szeretnénk ad^ni!
  \endverse

  \beginverse
    ^Az Úr, az Úr, ki a közösséget építi,
    És ^az Úr, az Úr, ki a közösséget építi,
    És ^az Úr, az Úr, ki a közösséget építi,
    És ^tovább szeretné ad^ni!
  \endverse
  
  \beginverse
    ^A szeretet, a szeretet, ki a közösséget építi,
    És ^a szeretet, a szeretet, ki a közösséget építi,
    És ^a szeretet, a szeretet, ki a közösséget építi,
    És ^tovább szeretné ad^ni!
  \endverse
  
  \beginverse
    ^Az Antioch, az Antioch, ki a közösséget építi,
    És ^az Antioch, az Antioch, ki a közösséget építi,
    És ^az Antioch, az Antioch, ki a közösséget építi,
    És ^tovább szeretné ad^ni!
  \endverse
  
  \beginverse
    ^És én, és te, és mi, az Úr, a szeretet, az Antioch, ki a közösséget építi,
    És ^én, és te, és mi, az Úr, a szeretet, az Antioch, ki a közösséget építi,
    És ^én, és te, és mi, az Úr, a szeretet, az Antioch, ki a közösséget építi,
    És ^tovább szeretné ad^ni!
  \endverse

\endsong
\beginsong{Indulj és menj}
[
  sr={Ezekiel 3,4},
  by={Hollósy Péter}
]

  \ifchorded
    \beginverse*
      {\nolyrics Előjáték: /: \[C] \[D] \[Em] \[H7]} :/ \rep{2}
    \endverse
  \fi

  \beginverse\memorize
    \[Em]Indulj és \[G]menj, \[D]hirdesd sza\[Em]vam,
    \[C]Népemhez \[D]küldelek \[Em]én! \[H7]
    \[Em]Tövis és \[G]gaz, \[D]vér és pa\[Em]nasz,
    \[C]Meddig hall\[D]gassam még \[Em]én?
  \endverse

  \beginchorus
    \[G]Küldelek \[D]én, \[H7]megáldlak \[Em]én,
    Csak \[C]menj és \[D]hirdesd sza\[Em]vam! \[H7]
    \[G]Küldelek \[D]én, \[H7]megáldlak \[Em]én,
    Csak \[C]menj és \[D]hirdesd sza\[Em]vam!
  \endchorus

  \beginverse
    ^Tüzessé ^teszem ^ajkai^dat,
    ^Gyémánttá ^homloko^dat. ^
    ^Népemnek ^őrévé ^rendellek ^én,
    ^Lelkemet ^adom mel^léd!
  \endverse

  \beginverse*
    Refrén
  \endverse

\endsong

\beginsong{Jézus néz rám}[]

  \ifchorded
    \beginverse*
      {
        \nolyrics Előjáték:
        \[C] \[Dm] \[Am] \[E7] \[Dm] \[G7] \[C] \[G7]
        \[C] \[Dm] \[Am] \[E7] \[Dm] \[G7] \[C]
      }
    \endverse
  \fi

  \beginverse
    \[C]Jézus \[Dm]néz rám a \[Am]két szemed\[E7]ből, arra \[Dm]kér, arra \[G7]vár, hogy megért\[C]sem, \[G7]
    \[C]Jézus \[Dm]néz rám a \[Am]két szemed\[E7]ből, arra \[Dm]kér, arra \[G7]vár, hogy szeres\[C]sem.
  \endverse

  \beginchorus
    /: \[C]Jézus \[H7]arra \[E]kér, mert \[C]bennünk \[H7]emberekben \[E]él, hogy
    \[C]meglás\[C7]sam és \[F]szeressem, ha \[C]eljön hozzám \[G7]valaki\[C]ben. \[G$^{\;(csak\;ismétlésnél)}$]{\ifchorded\hspace{0.4cm}\fi:/ \rep{2}}
    % vagy inkább \quad ?
  \endchorus

\endsong

\beginsong{Jöjj, itt az idő}[]

  \ifchorded
    \beginverse*
       {\nolyrics Előjáték: \[D] \[D\⁴] \[D] \[D\⁴]}
    \endverse
  \fi

  \beginverse\memorize
    \[D]Jöjj, itt az idő, hogy \[D\⁴]éb\[D]redj!
    \[A]Jöjj, itt az idő, hogy \[Em]áld\[F#m]juk \[G]Őt!
    \[D]Jöjj, szívedet add az \[D\⁴]Úr\[D]nak!
    \[A]Jöjj, imádd az Istent \[Em]úgy, \[F#m]ahogy \[G]vagy!
  \endverse

  \beginverse*
    Hát \[D]jö\[D\² D\⁴ D]jj!
  \endverse

  \beginchorus
    /: \[G]Minden nyelv megv\_{a}llja, hogy \[D]Ő az Úr! \[G]És mindenki t\_{é}rdet \[D]hajt!
    De a \[G]legnagyobb kincs mégis \[Hm]azokra vár, kik \[Em]most Őt válasz\[A\⁴]tják! \[A] :/ \rep{2}
  \endchorus

  \beginverse*
    1. versszak
  \endverse

  \beginverse*
    Hát \[D]jöjj! \[G] Ó \[D]jöjj! \[G] Hát \[D]jöjj!
  \endverse

\endsong

\beginsong{Kicsiny kis fényemmel}[]

  \beginverse\memorize
    \[G]Kicsiny kis fényemmel világítani fogok.
    \[C]Kicsiny kis fényemmel világítani fo\[G]gok.
    Kicsiny kis fényemmel \[H7]világítani fo\[Em]gok.
    Áldom \[D]Őt minden \[D7]nap és minden\[G]hol. \[C] \[G]
  \endverse

  \beginverse
    ^Elrejtsem-e fényemet? Nem! Világítani fogok.
    ^Elrejtsem-e fényemet? Nem! Világítani fo^gok.
    Elrejtsem-e fényemet? Nem! ^Világítani fo^gok.
    Áldom ^Őt minden ^nap és minden^hol. ^ ^
  \endverse

  \beginverse
    A ^Sátán sem állíthat meg! Nem! Világítani fogok.
    A ^Sátán sem állíthat meg! Nem! Világítani fo^gok.
    A Sátán sem állíthat meg! Nem! ^Világítani fo^gok.
    Áldom ^Őt minden ^nap és minden^hol. ^ ^
  \endverse

  \beginverse
    ^Így teszek, míg Jézus jön! Igen! Világítani fogok.
    ^Így teszek, míg Jézus jön! Igen! Világítani fo^gok.
    Így teszek, míg Jézus jön! Igen! ^Világítani fo^gok.
    Áldom ^Őt minden ^nap és minden^hol. ^ ^
  \endverse

\endsong

\beginsong{Király vagy}[]

  \beginverse*
    /: \[D]Király vagy, \echo{Király vagy}
    \[A]Király vagy, \echo{Király v\[Hm]agy}
    Jézus Ki\[G]rály. :/ \rep{2}
  \endverse

  \beginverse*
    /: Most \[D]felemeljük a szívünket,
    Most \[A]felemeljük kezeinket,
    \[Hm]Trónod elé járulunk, imádva \[G]légy! :/ \rep{2}
  \endverse

\endsong

\beginsong{Köszönöm, Jézus}[]

  \beginverse\memorize
    \[D] Köszönöm, J\_{é}zus, köszönöm, J\_{é}zus,
    Köszönöm, U\[G]ram, hogy szeretsz en\[D]gem!
    Köszönöm, \[F#m]J{\_{é}}zus, köszönöm, \[Hm]J{\_{é}}\[G]zus,
    Köszönöm, U\[D]ram, \[A] hogy szeretsz en\[D]gem! \[G] \[D]
  \endverse

  \beginverse
    ^ A Golgotára ment, és meghalt \_{é}rtem;
    Köszönöm, U^ram, hogy szeretsz en^gem!
    A Golgo^tára ment, és meghalt ^{\_{é}}r^tem;
    Köszönöm, U^ram, ^ hogy szeretsz en^gem! ^ ^
  \endverse

  \beginverse
    ^ A harmadik napon Ő f\_{e}ltámadott;
    Köszönöm, U^ram, hogy szeretsz en^gem!
    A harma^dik napon Ő f\_{e}l^táma^dott;
    Köszönöm, U^ram, ^ hogy szeretsz en^gem! ^ ^
  \endverse

  \beginverse
    ^ Ó, allel\_{u}ja, ó, allel\_{u}ja!
    Köszönöm, U^ram, hogy szeretsz en^gem!
    Ó, alle^l\_{u}ja, ó, alle^l{\_{u}}^ja!
    Köszönöm, U^ram, ^ hogy szeretsz en^gem! ^ ^
  \endverse

\endsong

\beginsong{Krisztus él bennem}[]

  \ifchorded
    \beginverse*
       {\nolyrics Előjáték: \[G] \[D] \[C] \[C-D]}
    \endverse
  \fi

  \beginchorus
    \[G]Krisztus él bennem, \[D]szívem virágzik.
    A \[Em]sötét nem győz, hát \[C]nem kell pánik.
    \[G]Jézus \[D]életét \[C]adta, hogy élj! \[D]

    \vspace{0.3cm}

    \[G]Krisztus él benned, \[D]ne légy lekvár,
    Hisz \[Em]oly sok kincs van, mi \[C]téged is vár!
    Hát \[G]kelj fel, \[D]indulj és \[C]Krisztussal járj!

    \vspace{0.3cm}

    /: \[D]Na-na-na, \[G]na na na \[D]na, na-na-na, \[Em]na na na \[C]na,
    na-na-na, \[G]na na na, \[D]na na na, \[C]na na! :/ \rep{2} \[D$^{\;(a\;legvégén\;nincs)}$]
  \endchorus

  \beginverse\memorize
    Az \[Em]Örömhír fénye \[C]ragyogjon éle\[D]tedben!
    A \[G]szeretet tüze \[Am]égjen a szíved\[D]ben!
    Hogy \[Hm]lássa meg a vi\[Em]lág, te is \[C]mondd velünk, \[Am]éld meg és add to\[D]vább! \[D4] \[D]
  \endverse

  \beginverse*
    Refrén
  \endverse

  \beginverse
    ^Hiába háborog ^alattunk a ten^ger,
    ^Jézus szemébe ^nézve félnünk ^nem kell.
    Ha ^Ő bennünk ^él, előttünk ^világos, ^ tiszta a ^cél. ^ ^
  \endverse

  \beginverse*
    Refrén
  \endverse

\endsong

\beginsong{Látod, újra este van}[]

  \beginverse\memorize
    \[C] Látod, újra este van, Hozzád száll most \[C/H]halk \[Am]szavam,
    \[Dm] Mindenségnek Iste\[F]ne, \[G] Téged dicsér éne\[C]kem.
  \endverse

  \beginverse
    ^ Köszönöm ezt a napot, annyi szépet s ^jót ho^zott.
    ^ Köszönöm a napsü^té^st, fényét szórta szerte^szét.
  \endverse

  \beginchorus
    \[C]Kérlek, ó jó Uram, \[Dm]add meg, Fényeddé \[F]váljak, jeled le\[C]hessek!
    \[C]Kérlek, Szentlelked \[Dm]küldd el, Ő ve\[F]zessen \[G]végig élete\[C]men!
  \endchorus

  \beginverse
    ^ Uram, mindent köszönök, bánatot é^s örö^möt,
    ^ Legyen rajtam szent ke^ze^d, óvja, védje éle^tem!
  \endverse

  \beginverse
    ^ Szeretném, ha szeretnél, minden szavad ^érte^ném,
    ^ Életemben követ^né^m, s végül Hozzád érkez^ném.
  \endverse

  \beginverse*
    Refrén
  \endverse
  
  \beginverse*
    1. versszak \ifchorded{\nolyrics (Csak lehúzásokkal.)}\fi
  \endverse

\endsong
\beginsong{Loyolai Szent Ignác imája}[]

  \ifchorded
    \beginverse*
       {\nolyrics Előjáték: \[C] \[G] \[Am] \[F] \[C] \[G] \[C] \[G7]}
    \endverse
  \fi

  \beginverse\memorize
    \[C]Fogadd el, \[G]Uram \[Am]szabadságo\[F]mat, \[C]fogadd el \[G]egé\[C]szen! \[G]
    \[C] Vedd értel\[G]memet, \[Am]akarato\[F]mat, és \[C]emlé\[G]kezé\[C]sem!
  \endverse

  \beginverse
    ^Mindazt, amim ^van ^és ami ^vagyok, ^Te adtad ^in^gyen.^
    ^Visszaadok, ^Uram ^visszaa^dok ^egyszerre ^min^dent.
  \endverse
  
  \beginchorus
    \[Am]Legyen \[E7]fölöttük \[Am]korlátlan \[F]úr \[C]rendel\[G]kezé\[C]sed!
    Csak \[Am]egyet \[E7]hagyj meg \[Am]ajándéko\[D7]dul: \[G]szeretnem \[F]Té\[Em]ged! \[G7]
  \endchorus

  \beginverse
    ^Csak a ^szeretet ^maradjon e^nyém a ^kegye^lem^mel, ^
    S ^minden, de ^minden ^gazdagság e^nyém, más ^semmi ^nem ^kell!
  \endverse

\endsong
\beginsong{Minden, mi él}
[
  li={Kék könyv / 153.}
]

  \ifchorded
    \versesep=12pt plus 3pt minus 7pt
  \fi

  \beginchorus
    \[D]Minden, mi él, csak Téged h\_{i}rdet,
    \[Hm]Minden dicsér, mert mind a m\_{ű}ved,
    \[G]Azzal, hogy él, ezt zengi N\_{é}ked:
    \[A]Dicsérlek én, \[A7]dicsérlek T\_{é}g\_{e}d!
  \endchorus

  \beginverse\memorize
    \[D]Dicsér az ég, Nap, Hold és csillagok,
    \[Hm]Fény és sötét, nap, éj és hajnalok.
    \[G]Dicsér a szél, felhő és hóvihar,
    \[A]A víz s a tűz, \[A7]megannyi tiszta dallal!
  \endverse

  \refrain

  \beginverse
    ^Dicsér a föld, dicséri szent neved,
    ^Mint jó anyánk, táplál s ad eledelt.
    ^Virág, gyümölcs, zöld fű, fa, hegyvidék,
    ^Tó és folyó, ^síkság és büszke b\_{é}rc!
  \endverse

  \refrain

  \beginverse
    ^A nagy világ létével énekel,
    ^Szavunkra vár, hogy hangja dal legyen.
    ^Zengjük tehát ég és Föld énekét,
    ^Zengjük velük: ^nagy Isten áldott l\_{é}gy!
  \endverse

  \refrain{\echo{\[D]Dicsérlek én!}}

\endsong

\beginsong{Minden nap}
[
  by={Pintér Béla},
  index={Mit is mondhatnék}
]

  \ifchorded
    \beginverse*
      {\nolyrics Előjáték: \[D] \[A] \[F#m] \[E]}
    \endverse
  \fi

  \beginverse\memorize
    \[D] Mit is \[A]mondhatnék, \[F#m] Te vagy, ki \[E]megváltottál,
    \[D] Drága \[A]véreddel \[F#m] hófe\[E]hérre mostál,
    \[D] Minden\[A]em lettél, \[F#m] Tiéd az \[E]egész lényem,
    \[D] Ég a \[A]tűz bennem, \[F#m] csak \[E]Érted élek \[A]m{\_{á}}r! \[D] \[F#m] \[E]
  \endverse

  \ifchorded
    \beginverse*
      {\nolyrics Közjáték: \[A] \[D] \[F#m] \[E]}
    \endverse
  \fi

  \beginverse
    ^ Minden ^nap, Uram, ^ ráál^lok igédre,
    ^ Téged ^hívl\_{a}k, ^ légy a ^segítségem,
    ^ Vezess ^át engem, ^ tovább a ^keskeny úton,
    ^ Add, hogy ^életem ^ a világban ^tüzet gy\_{ú}jt^{s\_{o}}n! ^ ^ ^
  \endverse

  \beginchorus
    \[A] Minden \[D]nap csak \[F#m]Érted \[E]élek.
    \[A] Minden \[D]nap kö\[F#m]vetlek \[E]Téged.
    \[A] Minden \[D]nap Ve\[F#m]led já\[E]rok, U\[A]ram. \[D] \[F#m] \[E]
  \endchorus

  \beginverse*
    /: \[A]Érted \[D]élek \[F#m]minden \[E]n{\_{a}}p! :/ \rep{3}
    U\[A]ram! \[D] \[F#m] \[E]
  \endverse

  \beginverse*
    Refrén \rep{2} \[A]
  \endverse

\endsong

\beginsong{Most nem sietek}
[
  by={Túrmezei Erzsébet}
]

  \beginverse\memorize
    \[Em]Most nem sie\[G]tek, most nem roha\[D]nok,
    \[Em]Most nem terve\[G]zek, \[C]most nem aka\[D]rok,
  \endverse

  \beginchorus
    \[G]Most nem teszek \[D]semmit sem,
    Csak \[C]engedem, hogy \[D]szeressen az \[G]Isten.
    \[G]Most nem teszek \[D]semmit sem,
    \[G]Csak engedem, hogy \[H7]szeressen az \[Em]Isten.
  \endchorus

  \beginverse
    ^Most megnyug^szom, most elpihe^nek,
    ^Békén, szaba^don, ^mint gyenge gye^rek,
    Nem, nem teszek ...
  \endverse

  \beginverse
    ^S míg ölel a ^fény, és ölel a ^csend,
    ^És árad be^lém, ^és újjáte^remt,
    Míg nem teszek ...
  \endverse

  \beginverse
    ^Új gyümölcs te^rem, másoknak te^rem,
    ^Érik csende^sen ^erő, győze^lem,
    Ha nem teszek ...
  \endverse

\endsong

\beginsong{Mustármag}[]

  \beginverse*
    /: Hogyha csak \[Am]mustármagnyi hitetek is \[E]volna...
    Ezt mondotta a mi U\[Am]runk. :/ \rep{2}
  \endverse

  \beginverse*
    /: És ha így \[Dm]szóltok a nagy he\[Am]gyekhez:
    Mozdulja\[E]tok, mozdulja\[Am]tok! :/ \rep{2}
  \endverse

  \beginverse*
    /: \[Am] Akkor a hegyek megindul\[E]nak,
    Megindulnak, megindul\[Am]nak. :/ \rep{4}
  \endverse

  \beginverse*
    /: \[Am]Jöjj el, jöjj el, ó \[E]jöjj el, Szent\[Am]lélek! :/ \rep{4}
  \endverse

\endsong

\beginsong{Napfivér, Holdnővér}[]

  \ifchorded
    \beginverse*
      {\nolyrics Előjáték: \[D4] \[D] \[D4] \[D]}
    \endverse
  \fi

  \beginverse\memorize
    \[D]Fivérem \[F#m]Nap és \[G]n{\_{ő}}\[A]vérem \[D]Hold,
    \[Hm] Oly' ritkán \[F#m]látlak, s \[G]hallom \[A]hango\[D]tok.
    \[Hm] Nyomaszt a \[F#m]sok \[G]gyötre\[A]lem és \[D]gond.
  \endverse

  % \ifchorded
  %   \beginverse*
  %     {\nolyrics Közjáték: \[Hm] \[F#m] \[G] \[A] \[D]}
  %   \endverse
  % \fi

  \beginverse
    ^Fivérem ^Szél és ^Leve^{g\_{ő}}^ég,
    ^ Nyisd ki sze^mem, hogy ^lássam, ^ami ^szép!
    ^ Körülö^lel a ^ragyo^gás, dics^fény.
  \endverse

  \beginchorus
    \[Hm]Mert Isten \[F#m]m{\_{ű}}ve \[G]minden \[A]teremt\[D]mény,
    \[Hm] Érzem jó\[F#m]ságát, és \[G]szívem újra {\[A4 A]é}l!
  \endchorus

  \beginverse
    ^Fivérem ^Nap és ^n{\_{ő}}^vérem ^Hold,
    ^ Most végre ^látok, s ^hallom ^hango^tok!
    ^ Megölel^ném az ^egész ^vilá^got!
  \endverse

  \refrain

  \beginverse*
    ^Fivérem ^Nap és ^n{\_{ő}}^vérem ^Hold,
    ^ Most végre ^látok, s ^hallom ^hango^tok!
    ^ Megölel^ném az ^egész ^vilá^got!
    % ^ Megölel^ném az ^egész ^vilá\[Gm]got! \[Gm7] ^
  \endverse

\endsong

\beginsong{Ne félj, mert megváltottalak}[]

  \beginchorus
    /: \[D]Ne félj, mert \[G]megváltotta\[A]lak,
    Neveden \[D]szólította\[Hm]lak,
    Karja\[Em]imba zárta\[A]lak,
    Örökre \[A7]enyém \[D]vagy! :/ \rep{2}
  \endchorus

  \beginverse\memorize
    \[A]Viruló réteken \[D]át \[G]hűs forrás fe\[A]lé vezetlek,
    \[Hm]Pásztorod va\[Em]gyok, elveszni \[A]senkit nem ha\[D]gyok.
    Karom fe\[Hm]léd t\_{á}\[Em]rom, kiárad \[G]áld\_{á}\[D]som.
    \[Hm]Nem rejtőzöm \[Em]el, szívem a \[A]szívednek fe\[D]lel,
    Amikor \[Hm]úgy \_{é}r\[Em]zed, nyomaszt a \[A7]{\_{é}}\[D]let.
  \endverse

  \beginverse*
    Refrén
  \endverse

  \beginverse
    ^Nem taszítalak ^el, ^amikor vétke^zel,
    Irgalmat ^lelsz a szívem^ben, örök fe^léd a hűsé^gem,
    Amerre ^jársz, v\_{é}^dlek, nyomodba ^l{\_{é}}^pek.
    ^Nem rejtőzöm ^el, szeretet^lángom átö^lel,
    Ne félj, ha ^éjb{\_{e}}n ^jársz, hidd, hogy a ^fény v\_{á}r ^rád!
  \endverse

  \beginverse*
    Refrén
  \endverse

\endsong

\beginsong{Nézz, testvér, fel - !!}[]

  \beginchorus
    \[D]Nézz, testvér, fel, az \[F#m]Úr van itt, \[G]lángol\[A]jon a szí\[D]vünk!
    \[G]Közel van már \[F#m]üdvös\[Hm]ségünk,
    \[G]Jöjj el \[Em]Jézu\[A]sunk, \[G]jöjj el \[A]Jézu\[D]sunk!
  \endchorus

  \beginverse\memorize
    \[D]Kerestem arcodat, U\[F#m]ram, leha\[G]joltál \[A]énhoz\[D]zám.
    \[D]Félelmek, bűnök tép\[F#m]tek, de Te \[G]meggyó\[A]gyítot\[D]tál.
  \endverse

  \beginverse*
    Refrén
  \endverse

  \beginverse
    ^Figyelme szeretőn kí^sér, meghall^gatja ^az i^mám.
    ^Gyötrelmeimben vigaszt nyújt, ezért ^áldja ^Őt a ^szám!
  \endverse

  \beginverse*
    Refrén
  \endverse

  \beginverse
    ^Jöjjön, ki szomjas és i^gyék, aki ^hisz, ^élni ^fog!
    ^Szívének rejtett mélyé^ből az é^lő víz ^felbu^zog.
  \endverse

  \beginverse*
    Refrén
  \endverse

\endsong

\beginsong{Nincs más isten}
[
  by={Chris Tomlin, fordította Váradi Attila},
  li={http://nyugodtleszazeleted.blogspot.hu/2010/11/nincs-mas-isten.html},
  index={Nálad lett borrá a víz}
]

  \ifchorded
    \beginverse*
      {\nolyrics Előjáték: /: \[Em7] \[Cadd2] \[G] \[D/F#] :/ \rep{2}}
    \endverse
  \fi

  \beginverse\memorize
    \[Em]Nálad lett \[C]borrá a \[Em]víz, \[C] h\[Em]oltakat \[C]életre \[G]hívsz,
    Nincs más \[Am7]isten, nincsen \[D]más.
  \endverse

  \beginverse
    ^Te ragyo^god be az ^éjt, ^ ^Te hozod ^el a re^ményt.
    Nincs más ^isten, nincsen ^más.
  \endverse

  \beginchorus
    \[Em]Te vagy az Isten, \[C]aki nevére \[G]leborul minden a \[D]Földön, az égen,
    \[Em]Hegyeket mozdít, \[C]beteget gyógyít az \[G]Úr ma i\[D]s!

    \[Em]Te vagy az Isten, \[C]aki nevére \[G]leborul minden a \[D]Földön, az égen,
    \[Em]Szíveket hódít, \[C]sebeket gyógyít az \[G]Úr ma i\[D]s!

    \ifchorded
      \vspace{0.3cm}
      {\nolyrics Közjáték: /: \[Em7] \[Cadd2] \[G] \[D/F#] :/ \rep{2}}
      \vspace{0.3cm}
    \fi

    /: \[Em] És hogyha Isten velünk, \[C] ember mit árthat nekünk,
    \[G] Mert hogyha Te vagy velünk, \[D] ki lehet ellenünk? :/ \rep{2}
  \endchorus

  \beginverse*
    Refrén \echo{\[D]Mi állhat ellenünk?}
  \endverse

\endsong

\beginsong{Oly' jó áldani}[]

  \beginverse*
    \[F]Szívemből \[C]ünnepi \[F]ének á\[C]rad,
    \[G]Háladalt \[G7]éneklek az \[F]ég Urá\[C]nak!
    \[G]Ujjongva \[G7]tör fel a \[F]hála ben\[C]nem,
    \[Dm]Jósága körülvesz \[G]szárnyként \[G7]en\[C]gem. \[Dm] \[G] \[C]
  \endverse

  \beginverse*
    \[G]Adjatok \[G7]hálát az \[C]Istennek!
    \[G7]Adjatok hálát az \[C]Istennek!
    \[C7]Ő a mi teremtő \[F]Kirá\[G]lyunk,
    \[D]Alle-allelu\[G]ja, \[F]allelu\[C]ja, \[F]allelu\[C]ja! \[H7]
  \endverse

  \beginverse*
    /: \[E]Hála legyen, hála legyen, hála! \[A]Áldott az \[E]Úr! :/ \rep{3}
  \endverse

  \ifchorded
    \beginverse*
      {\nolyrics Átvezető: \[D] \[A] /: \[A] \[E] \[A] \[E] :/ \rep{2}}
    \endverse
  \fi

  \beginverse\memorize
    \[A]Oly' jó ál\[E]dan\[A]i az \[E]Ur\[A]at, \[A] \[E] \[A] \[E]
    \[E]Ujjongó\[A]an \[E]jó Is\[A]ten\[E]nek zen\[A]ge\[D]ni zsoltáro\[A]kat. \[A] \[E] \[A] \[E]
  \endverse

  \beginverse
    ^Már kora reg^ge^len jó^sá^gát, ^ ^ ^ ^
    És ^éjszaká^kon ^át hir^de^tem szün^te^len nagy irgal^mát. ^ ^ ^ ^
  \endverse

  \beginverse
    ^Pengő hang^sze^ren, lan^to^kon, ^ ^ ^ ^
    ^Ujjongó^an ^jó Is^ten^nek zen^ge^ni hála^szót. ^ ^ ^ ^
  \endverse

  \beginverse
    ^Az öröm hang^ja^i szól^ja^nak, ^ ^ ^ ^
    ^Gitár húr^ja^in magasz^ta^lom uj^jong^va csodái^dat! ^ ^ ^ ^
  \endverse

  % \ifchorded
  %   \chordsoff
  % \fi

    \beginverse
      ^Uram, jó voltál énhozzám,
      Életemben nagy dolgokat műveltél sok év során.
    \endverse

    \beginverse
      Sziklám, oltalmam Te voltál,
      Oly' irgalmasan jó, gondoskodó Istenem, égi Atyám.
    \endverse

  % \ifchorded
  %   \chordson
  % \fi

  \beginverse
    ^Hála, di^cső^ség Ne^ved^nek, ^ ^ ^ ^
    ^Most és ö^rök^ké az ^é^gen, a ^Föl^dön! \_{A}^men. ^ ^ ^ ^ \[A]
  \endverse

\endsong

\beginsong{Ott lélegzel a fákban}[]

  \ifchorded
    \beginverse*
       {\nolyrics Előjáték: \[D] \[A] \[A7] \[D]}
    \endverse
  \fi

  \beginverse\memorize
    \[D]Ott lélegzel a \[A]fák\[D]ban, \[G]ott mosolyogsz a vi\[A]rágban,
    \[A]Ott vagy a szélben, a \[D]frissben, ó, boldog, boldog \[A7]Is\[D]ten!
  \endverse

  \beginverse
    ^Szép sorsomat Te ^sző^tted, ^jó leborulni e^lőtted,
    ^Te vagy az útnak a ^vége, szent mélység, áldott ^bé^ke!
  \endverse

  \beginverse*
    1. versszak
    2. versszak
  \endverse

\endsong
\beginsong{Ő az Úr}
[
  by={Marvin Frey}
]

  \ifchorded
    \versesep=12pt plus 3pt minus 8pt
  \fi

  \beginverse\memorize
    Ő az \[H7]Úr! Ő az \[E]Úr! Meghalt értem a ke\[C#m]reszten, Ő az \[H7]Úr!
  \endverse

  \beginchorus
    Minden \[E]térd megha\[E7]joljon,
    Minden \[A]nyelv csak Róla \[F#m]szóljon,
    Mert \[E]Jézus, \[H7]Ő az \[E A E]Úr!
  \endchorus

  \beginverse
    Ő az ^Úr! \echo{Ő az Úr!} Ő az ^Úr! \echo{Ő az Úr!}
    Feltámadott a ha^lálból, Ő az ^Úr!
  \endverse

  \refrain

  \beginverse
    Ő az ^Úr! \echo{Ő az Úr!} Ő az ^Úr! \echo{Ő az Úr!}
    Betölt engem Szentlel^kével, Ő az ^Úr!
  \endverse

  \refrain

  \beginverse
    Ő az ^Úr! \echo{Ő az Úr!} Ő az ^Úr! \echo{Ő az Úr!}
    Táplál engem szent tes^tével, Ő az ^Úr!
  \endverse

  \refrain

  \beginverse
    Ő az ^Úr! \echo{Ő az Úr!} Ő az ^Úr! \echo{Ő az Úr!}
    Táplál engem szent vé^rével, Ő az ^Úr!
  \endverse

  \refrain

  \beginverse*
    Ő az ^Úr! \echo{Ő az Úr!} Ő az ^Úr! \echo{Ő az Úr!}
    Ő az \[H7]Úr! \echo{Ő az Úr!} Ő az {\[A E]Ú}r! \echo{Ő az Úr!}
  \endverse

\endsong

\beginsong{Péter, ne sírj}[]

  \beginverse\memorize
    \[Em]Sírásodat hagyd nyugodtan abba,
    A kapuk mögött vissza\[Em/F#]állt a \[G]rend!
    \[Am]Ne sírj, oly' lágyan jár a \[Em]szél
    \[C]Róma s a \[D]Via Appia fe\[Em]lett!
  \endverse

  \beginchorus
    \[Am]Péter, ne \[Em]sírj, mert \[H7]Jézus megbo\[Em]csát,
    \[Am]Péter, ne \[Em]sírj, hisz' \[H7]tudod, hogy megbo\[Em]csát!
  \endchorus

  \beginverse
    ^Gyengeségünk néha úgy esik ránk,
    Mint kisgyermekre esik a ^féle^lem.
    ^Gyengeségért mindig bünte^tés jár,
    De ^büntetése ^nem sújt, fele^mel!
  \endverse

  \beginverse*
    Refrén
  \endverse

  \beginverse
    ^Reménytelen a bűn után az élet,
    Hogy mégis élsz már alig ^hiszed ^el,
    ^Alig hiszed, hogy nem lett akkor ^vége,
    Csak ^egyet tudsz, a ^hulló könnye^ket!
  \endverse

  \beginverse*
    Refrén \rep{2}
  \endverse

\endsong

\beginsong{Rajta, dicsérjétek az Úr nevét}[]

  \ifchorded
    \beginverse*
      {\nolyrics Előjáték: \[G-C] \[G!]}
    \endverse
  \fi

  \beginchorus
    Rajta, di\[G]csérjé\[C]tek az Úr ne\[G]vét,
    Ti \[C]mindnyájan, kik \[G]hűen \[Em]szolgáljátok \[D]Őt! \[D\⁴] \[D]
    Áldjátok \[G]hát há\[C]romszor szent ne\[G]vét,
    Ma\[C]gasztaljátok \[G]ének\[Am]szóval \[D]örök\[G]ké! \[C] \[G]
  \endchorus

  \beginverse\memorize
    Igen, tu\[Em]dom, Istenünk hatal\[C]mas,
    Végbevisz \[Am]mindent, amit csak a\[D]k\[H\⁷]ar.
    A hegyek\[Em]től a tenger mélyé\[C]ig,
    A földtől az \[Am]égbolt magasságá\[D]ig. \[D\⁴] \[D]
  \endverse

  \refrain

  \beginverse
    Lássátok ^hát, mily' irgalmas hoz^zánk,
    Jóságos ^Úr és hűséges Ki^r^ály.
    Énekel^jétek hát az Ő ne^vét,
    Istent di^csérje minden teremt^mény! ^ ^
  \endverse

  \refrain[2]

\endsong

\beginsong{Rejts most el}
[
  by={Reuben Morgan, fordította Labadics Kriszta}
]

  \beginverse\memorize
    \[C]Rejts \[G/H]most \[Am]el a \[F]szár\[D/F#]nyad a\[G]lá,
    E\[C]rős \[G]kéz\[F]zel \[Dm]takarj \[D]be en\[G]gem!
  \endverse

  \beginchorus
    \[C]Tenger tombol, \[F]zúg, sü\[G]vít a sz\[C]él,
    \[C]Te emelsz fel \[F]a vi\[G]har fö\[Am]lé!
    \[C]Uralkodsz hul\[F]lámok \[G]habja\[C]in,
    \[C]Szívem nem f\[F]él, ben\[G]ned re\[Am]mél!
    \[C]Szívem nem f\[F]él, ben\[G]ned re\[F]mél! \[G]
  \endchorus

  \beginverse
    Csak ^Is^ten^ben bízz, ^én ^lel^kem!
    ^Mert ^ő ^él, ^nagyobb ^minden^nél!
  \endverse
  
  \beginverse*
    Refrén
  \endverse

\endsong
\beginsong{Szentlélek, jöjj}[]

  \versesep=12pt plus 3pt minus 12pt

  \ifchorded
    \beginverse*
       {\nolyrics Előjáték: \[Em] \[D] \[Hm] \[Em] \[D] \[Hm] \[Em]}
    \endverse
  \fi

  \beginchorus\memorize
    \[Em]Szentlélek, jöjj, lobogó \[D]Láng!
    Szentlélek, jöjj, \[Hm]a világ \[Em]vár!
    Szentlélek, jöjj, viharos \[D]szél!
    Jöjj, \[Hm]áradj \[Em]szét!
  \endchorus

  \beginverse
    ^Jöjj el, Élő vízfor^rás,
    Jöjj, a szívünk ^Téged ^vár!
    Jöjj, ki fényt adsz lelkünk^nek,
    Jöjj, úgy ^várunk ^Rád!
  \endverse

  \beginverse*
    Refrén
  \endverse

  \beginverse
    ^Jöjj, igazság forrá^sa,
    Jöjj, imádunk ^mindnyá^jan!
    Jöjj, reményünk éleszd ^fel,
    Jöjj ke^gyelmed^del!
  \endverse

  \beginverse*
    Refrén
  \endverse

  \beginverse
    ^Jöjj, a néped gyűjtsd egy^be!
    Jöjj, az alvót ^ébreszd ^fel!
    Jöjj, a bűntől tisztíts ^meg,
    Báto^ríts min^ket!
  \endverse

  \beginverse*
    Refrén \rep{2}
  \endverse

\endsong

\beginsong{Szeretem örökké}
[
  index={Eléd lépek, jó Uram}
]

  \beginverse\memorize
    \[C]Eléd lépek, jó Uram, hol \[G/H]béke és nyugalom \[Am]vár,
    \[Em]Vágyom, hogy érints \[Dm]Lelkeddel, hogy \[F]múljék, ami \[G]fáj.
    Én \[C]hiszem, hogy itt vagy közöttünk és \[G/H]halkan, szelíden \[Am]hívsz.
    És mi \[Em]társaid leszünk \[Dm]örökké, hol \[F]boldog dalát \[G]zengi minden \[C]szív.
  \endverse

  \beginchorus
    /: \[Am]Áldom az Urat míg \[Em]élek, és \[F]szeretem örök\[C]ké,
    \[Am]Örömmel mondok \[Em]hálát a \[F]világon minden\[G]ért. :/ \rep{2}
  \endchorus

  \beginverse
    ^Közénk térdelsz, jó Urunk, ^féltő szeretet^tel,
    És egy ^új életnek ^boldogsága ^érző szívre ^lel.
    Én ^tudom, hogy itt vagy közöttünk és ^halkan, szelíden ^hívsz.
    És mi ^társaid leszünk ^örökké, hol ^boldog dalát ^zengi minden ^szív.
  \endverse

  \beginverse*
    Refrén \[C]
  \endverse

\endsong

\beginsong{Táncolj az Úrnak}
[
  by={M. \& M.-F. Penhard, Szulyovszky Endre, Szulyovszky Rita},
  li={Kék könyv / 201.}
]

  \ifchorded
    \versesep=12pt plus 3pt minus 10pt
  \fi

  \ifchorded
    \beginverse*
      {\nolyrics Előjáték: /: \[Em] \[D] \[Em] \[Hm] \[G] \[D] \[Em] \[Hm] \[Em] :/ \rep{2}}
    \endverse
  \fi

  \beginchorus
    /: \[Em]Táncolj az \[D]Úrnak, \[Em]dicsérjed ne\[Hm]vét!
    \[G]Ujjongj, hisz' \[D]Ő a te \[Em]Meg\[Hm]vál\[Em]tód! :/ \rep{2}
  \endchorus

  \beginverse\memorize
    Mint \[Em]Dávid az \[D]Úr lá\[Em]dája e\[Hm]lőtt,
    \[G]Táncol\[D]junk \[Em]Is\[Hm]ten\[Em]nek!
    Az \[Em]Ő or\[D]szága már \[Em]köztünk \[Hm]van,
    \[G]Daloljon \[D]ajkunk \[Em]uj\[Hm]jong\[Em]va!
  \endverse

  \refrain

  \beginverse
    Mint ^Mária ^Erzsébet ^házá^ban,
    ^Ujjong a ^bensőm ^há^lá^val.
    A ^Lélek a ^szívemet ^eltöl^ti,
    ^Alle^luja, ^így ^zen^gi!
  \endverse

  \refrain

  \beginverse
    Kik ^kedvesek az ^Úrnak, ^dicsérik ^Őt,
    ^Hűsé^gükkel ^szol^gál^nak.
    Mint ^betlehemi ^nyájak ^pásztora^i,
    ^Hódol^junk ^szí^ne e^lőtt!
  \endverse

  \refrain

  \ifchorded
    \beginverse*
      {\nolyrics Utójáték: /: \[Em] \[D] \[Em] \[Hm] \[G] \[D] \[Em] \[Hm] \[Em] :/ \rep{2}}
    \endverse
  \fi

\endsong

\beginsong{Terád vár egy szép ország}[]

  \beginverse\memorize
    Terád \[E]vár egy szép ország,
    Terád \[A]vár egy szép or\[E]szág,
    Terád vár egy szép ország, ahová me\[H7]gyek, ahová megyek.
    Terád \[E]vár egy szép or\[E7]szág,
    Terád \[A]vár egy szép or\[Am]szág,
    Terád \[E]vár egy szép or\[H7]szág, ahová me\[E]gyek, aho\[A]vá me\[E]gyek.
  \endverse

  \beginverse
    Nincs ott ^többé könnyezés,
    Nincs ott ^többé könnye^zés,
    Nincs ott többé könnyezés, ahová me^gyek, ahová megyek.
    Nincs ott ^többé könnye^zés,
    Nincs ott ^többé könnye^zés,
    Nincs ott ^többé könnye^zés, ahová me^gyek, aho^vá me^gyek.
  \endverse

  \beginverse
    Isten ^szép országa ez,
    Isten ^szép országa ^ez,
    Isten szép országa ez, ahová me^gyek, ahová megyek.
    Isten ^szép országa ^ez,
    Isten ^szép országa ^ez,
    Isten ^szép országa ^ez, ahová me^gyek, aho^vá me^gyek.
  \endverse

  \beginverse
    Jézus ^Krisztus vár ott rád,
    Jézus ^Krisztus vár ott ^rád,
    Jézus Krisztus vár ott rád, ahová me^gyek, ahová megyek.
    Jézus ^Krisztus vár ott ^rád,
    Jézus ^Krisztus vár ott ^rád,
    Jézus ^Krisztus vár ott ^rád, ahová me^gyek, aho^vá me^gyek.
  \endverse

  \beginverse
    Add át ^néki az életed,
    Add át ^néki az éle^ted,
    Akkor Isten szép országát elnye^red, te is elnyered.
    Add át ^néki az éle^ted,
    Add át ^néki az éle^ted,
    Akkor ^Isten szép or^szágát elnye^red, te is ^elnye^red. \[A] \[E]
  \endverse

\endsong

\beginsong{Teremts bennem}
[
  li={Kék könyv / 208.}
]

  \beginverse*
    /: \[G]Teremts bennem \[D]tiszta szí\[C]vet, ó U\[G]ram!
    Az \[Em]erős lelket \[D]újítsd meg ben\[G-C]nem! \[G]
    \[G]Teremts bennem \[D]tiszta szí\[C]vet, ó U\[G]ram!
    Az \[Em]erős lelket \[D]újítsd meg ben\[G-C]nem! \[G] :/ \rep{2}
  \endverse

  \beginverse*
    /: \[Am]Ne vess el en\[D]gem a Te or\[G]cád e\[E]lől!
    \[Am]Szentlelked \[D]ne vondd meg tő\[G]lem! \[G7]
    \[Am]Támogass az \[D]engedelmes\[G]ség lelké\[Em]vel!
    \[Am7]Szabadításod \[D7]örömét add ne\[G]kem! \[G7] :/ \rep{2}
  \endverse

  \beginverse*
    \[Am7]Szabadításod \[D7]örömét add ne\[G-C]kem! \[G]
  \endverse

\endsong

\beginsong{Uram, Tehozzád futok}[]

  \beginverse*\memorize
    Uram, \[Em]Tehozzád fu\[D]tok, élő \[Hm7]vízre szomja\[Em]zom,
    Közel\[C]séged, ó mily \[D]jó énné\[Em]kem. \[Em!]
  \endverse

  \beginverse*
    Kérlek, ^ne menj el tő^lem, légy min^dig segítsé^gem,
    Úgy kí^vánlak Té^ged, Iste^nem. ^
  \endverse

  \beginverse*
    Én pedig \[G]szüntelen remél\[D]ek, egyre \[Hm7]jobban dicsér\[Em]lek,
    Ajkam \[Am7]beszéli a Te \[Am/F#]igazságo\[H4]dat. \[H]
    Hadd le\[G]gyen most a da\[D]lom jó i\[Hm]llat oltáro\[Em]don,
    Nagyon \[C]szeretlek Té\[D]ged, Jézu\[Em]som. \[Em!]
  \endverse

  \beginverse*
    \textnote{A legvégén:}
    Nagyon \[C]szeretlek Té\[D]ged,
    Nagyon \[C]szeretlek Té\[D]ged,
    Nagyon \[C]szeretlek Té\[D]ged, Jézu\[Em]som. \[Em!]
  \endverse

\endsong

\beginsong{Utad vár rád}
[
  index={Mélyen a lelkedben}
]

  \ifchorded
    \beginverse*
      {\nolyrics Előjáték: \[Dm-Dm\⁴-Dm] \[F-F\²-F] \[C-C\⁴-C] \[Dm!-A!]}
    \endverse
  \fi

  \beginverse\memorize
    \[Dm]Mélyen a \[A]lelkedben egy \[Dm]tiszta fo\[C]lyó,
    Amit \[F]áradni \[Gm]érzel, Ő a \[C]Mindenha\[Dm]tó.
    \[Dm]Isteni \[A]üzenet az \[Dm]éjben a \[C]f{\_{é}}ny,
    \[F]Királysága \[Gm]már a \[C]szívedben \[Dm]él!
  \endverse

  \beginchorus
    Utad \[F]vár \[C]rád, \[Gm]indulj \[Dm]hát!
    \[F]Nálad van a \[Gm]kincs, mire \[C]mindenki \[Dm]vágy,
    Utad \[F]vár \[C]rád, \[Gm]indulj \[Dm]hát!
    \[F]Benned az \[G]isteni \[Gm]tűz a Földre \[Dm]sz{\_{á}}ll.
  \endchorus

  \beginverse
    ^Reménységed ^forrása ^égi A^ty{\_{á}}d,
    ^Szeretetének ^nincs, ki ^{\_{e}}llen^áll.
    ^Te lehetsz a ^fénylő jel a ^hegy tete^j{\_{é}}n,
    ^Te lehetsz, ki ^elviszed az ^üzene^tét.
  \endverse

  \refrain[2]

\endsong

\beginsong{A vak ember}[]

  \ifchorded
    \beginverse*
      {\nolyrics Előjáték: /: \[Am] \[G] \[F] \[E] :/ \rep{2}}
    \endverse
  \fi

  \beginverse\memorize
    Az \[Am]úton a \[G]vak ember \[F]ül és ki\[E]ált,
    Az \[Am]úton a \[G]vak ember \[F]ül és ki\[E]ált,
    Az \[Am]úton a \[G]vak ember \[F]ül és ki\[E]ált:
  \endverse

  \beginchorus
    Oh, oh, \[E\⁷]oh!
    Hol van az {\[Am G F]{\_{ú}}}t? \[E]Hol van a f{\[Am G F]{\_{é}}}ny, \[E]az igazs{\[Am G F]{\_{á}}}g,
    Ami haza\[E]visz? Oh-\[E\⁷]oh-oh, \[E]oh-oh, \[E\⁷]oh-oh!
  \endchorus

  \beginverse
    A ^béna is ^az úton ^ül és ki^ált,
    A ^béna is ^az úton ^ül és ki^ált,
    A ^béna is ^az úton ^ül és ki^ált:
  \endverse

  \refrain

  \beginverse
    A \[Am]néma is \[G]az úton \[F]ül és (tátog),
    A \[Am]néma is \[G]az úton \[F]ül és (tátog),
    A \[Am]néma is \[G]az úton \[F]ül és \[E]kiált:
  \endverse

  \refrain

  \beginverse
    S mi ^mind csak ü^lünk és a ^szívünk ki^ált,
    S mi ^mind csak ü^lünk és a ^szívünk ki^ált,
    S mi ^mind csak ü^lünk és a ^szívünk ki^ált:
  \endverse

  \refrain

  \beginverse
    De \[Am]Jézus is \[G]ott ül ve\[F]lünk és ki\[E]ált,
    De \[Am]Jézus is \[G]ott ül ve\[F]lünk és ki\[E]ált,
    De \[Am]Jézus is \[G]ott ül ve\[F]lünk és ki\[E]ált:
    Oh, oh, \[E\⁷]oh!
    Én vagyok az {\[Am G F]{\_{ú}}}t, \[E]az igazs{\[Am G F]{\_{á}}}g, \[E]és az él{\[Am G F]{\_{e}}}t,
    Ami haza\[E]visz! Oh-\[E\⁷]oh-oh, \[E]oh-oh, \[E\⁷]oh-oh!
  \endverse

  \beginverse*
    /: \[Am]Jézus :/ \rep{2}
  \endverse

  \beginverse*
    Oh, oh, \[E\⁷]oh!
    Én vagyok az {\[Am G F]{\_{ú}}}t, \[E]az igazs{\[Am G F]{\_{á}}}g, \[E]és az él{\[Am G F]{\_{e}}}t,
    Ami haza\[E]visz! Oh-\[E\⁷]oh-oh, \[E]oh-oh, \[E\⁷]oh-oh!
  \endverse

  \beginverse*
    \[Am]Jézus
  \endverse

\endsong

\beginsong{A világnak Krisztus kell}
[
  by={II. János Pál pápa, Sillye Jenő}
]

  \ifchorded
    \beginverse*
      {\nolyrics Előjáték: \[E] \[H\⁷] \[E] \[A] \[E]}
    \endverse
  \fi

  \beginchorus
    /: A \[E]világnak \[H\⁷] Krisztus \[E]kell!
    A \[A]világnak \[E] Krisztus \[H\⁷]kell!
    A \[E]világnak \[G#]kellesz \[C#m]te is,
    \[A]Mivel te \[E]Krisztushoz \[H\⁷]tartoz\[E]ol! :/ \rep{2}
  \endchorus

  \beginverse*
    \[A]Emlé\[E]kezz! \[A]Emlé\[E]kezz, te \[E]Krisztushoz \[H\⁷]tartoz\[E]ol.
  \endverse

\endsong

\beginsong{Vizek felett \\ Oceans}
[
  by={Hillsong United}
]

  \ifchorded
    \beginverse*
      {\nolyrics Előjáték: \[Hm] \[D] \[A] \[G]}
    \endverse
  \fi

  \beginverse\memorize
    \[Hm] Meghívtál, hogy vízre \[D]lépjek, hol nélkü\[A]led elsüllye\[G]dek.
    \[Hm] Ebben megtalállak \[D]Téged, a mélység\[A]ben megt\_{a}rt hi\[G]tem.
  \endverse

  \beginchorus
    \[G] Nagy ne\[D]ved h\_{í}vom \[A]én \[A4] \[G] és felné\[D]zek a vizek fö\[A]lé.
    Ott l\_{á}tlak \[G]én. A lelkem \[D]benned m\_{e}gpi\[A]hen. Enyém v\[G]agy és \[A]én Ti\[Hm]ed.
  \endchorus

  \beginverse
    ^ A mélységnél nagyobb ke^gyelmed, mely elve^zet és tart en^gem.
    ^ Ha elbuknék és nagyon ^félnék, Te nem hagysz ^el és nem inogsz ^meg.
  \endverse

  \beginverse*
    Refrén
  \endverse

  \ifchorded
    \beginverse*
      {\nolyrics Közjáték: /: \[Hm] \[G] \[D] \[A] :/ \rep{2}}
    \endverse
  \fi

  \beginverse*
    /: \[Hm] Lélek add, hogy Benned \[G]teljesen megbízzak,
    A vízen \[D]bátran Veled járjak, és \[A]bárhová hívsz, menjek!
    \[Hm] Vigyél tovább, mint a \[G]lábam tudna menni,
    Taníts \[D]teljes hittel járni, jelen\[A]létedben élni! :/ \rep{3}
  \endverse

  \beginverse*
    Refrén \rep{2} \ifchorded{\nolyrics (Az elsőnek a végén \[D] -vel.)}\fi
  \endverse

  \beginverse*
    Enyém \[G]vagy és \[A]én Ti\[Hm]ed. Enyém \[G]vagy és \[A]én Ti\[D]ed.
  \endverse

\endsong


    \setcounter{songnum}{101}

\input{songs/alleluja-k243.tex}
\beginsong{Atya, áldott legyél}
[
  by={zene és szöveg: A. Fleury, alaptálta: Fábry K.}
]

  \beginchorus
    Atya, \[G]ál\[D]dott legy\[Em]él!
    Fiú, \[Am]téged i\[C]mádjon a \[Am]föld és az \[D]ég!
    Lélek \[G]á\[D]radj re\[Em]ánk!
    Örök \[Am]Iste\[D]nünk, Téged \[C]áld az i\[G]mánk.
  \endchorus

  \beginverse\memorize
    \[Em]Jó Aty\[C]ánk, örök \[D]Alkotónk, \[C]legyen meg az \[D]akara\[G]tod!
    Amint a \[Em]mennyben a te \[Am]szentje\[D]id, \[H7]dicsérünk \[Em]Téged itt a \[Am]földön \[D]is.
  \endverse

  \beginverse*
    Refrén
  \endverse

  \beginverse
    ^Jézu^sunk, a mi ^Pártfogónk, ^Te vagy a mi ^Szabadí^tónk,
    Egyszü^lött, kit nekünk ^ad az A^tya, ^jöjj, U^runk, ma hívunk: ^Marana ^tha!
  \endverse

  \beginverse*
    Refrén
  \endverse

  \beginverse
    ^Szállj le ^ránk, Lélek^istenünk! ^Az utunkon ^légy ve^lünk!
    Jöjj, ve^zess, Te vagy az ^Élte^tőnk! Ha ^fára^dunk, kérünk ^adj e^rőt!
  \endverse

  \beginverse*
    Refrén \rep{2}
  \endverse

\endsong

\beginsong{Atyaisten, a néped}
[
  by={zene és szöveg: A. Broeders, adaptálta: Fábry K.}
]

  % https://youtu.be/TuE9nzUoTQA

  \beginverse
    Atya\[G]isten a \[Hm]néped jön \[C]Hozz\[Em\⁷⁄H]ád,
    És \[Am]el\[G⁄H]hozza \[C\²]ál\[C]doza\[D\⁴]t\[D]át,
    Mely a \[G]munkánk s a \[Hm]földünk gyü\[C]mölcs\[Em]e:
    \[Am]A szín\[G⁄H]bor \[C]és \[D\⁷]a ke\[G]nyér.
  \endverse

  \beginverse
    Az ^oltárra ^tesszük most ^vél^ük
    A ^di^cséret ^ál^doza^t^át,
    Vigye ^angyalod ^Hozzád az ^égb^e
    ^Fölsé^ged ^szí^ne e^lé.
  \endverse

  \beginverse
    A Te ^Szentlelked ^által lesz ^ért^ünk
    Ez a ^ke^nyér és ^bor ^éte^l^ünk.
    Az ^Oltári^szentségben ^Kriszt^us
    ^Eljön, ^és ^üd^vö^zít.
  \endverse

\endsong

\beginsong{Bárányom, bárányom}[]

  \ifchorded
    \beginverse*
      {\nolyrics
        Előjáték: \[G] \[Am] \[Em] \[C] \[G-Em] \[Am-D] \[G] \[D]
        \hspace{1.42cm} \[G] \[Am] \[Em] \[C] \[G-Em] \[Am-D] \[G]
      }
    \endverse
  \fi

  \beginverse\memorize
    \[G]Bárányom, \[Am]Bárányom, \[Em]Istennek B\_{á}\[C]ránya,
    \[G]Könyörülj \[Em]rajtam, \[Am]hadd jussak \[D]jobb sors\[G]ra! \[D]
    \[G] Adj nekem \[Am] nagy hitet, \[Em] amíg le\[C]{h\_{e}}t!
    \[G]Formáld \[Em]kedved sze\[Am]rint a \[D]szív\_{e}\[G]met!
  \endverse

  \beginchorus
    Én U\[Em]ram, add meg nekem,
    Jó U\[Am]ram, add meg nekem,
    \[D]Hadd legyen Mel\[D\⁷]letted majd a \[G]helyem!
    Lesem \[Em]minden vágyadat, tiszte\[Am]lem házadat.
    Én U\[F#]ram, jó Uram, add meg ne\[Am]k\[D]{\_{e}}m!
  \endchorus

  \beginverse
    ^ Én Uram, ^ adj egy kis ^ időt ne^kem,
    ^Vékony kis ^szalm\_{a}^szál ^az én hi^tem! ^
    ^ De ha Te ^ segítesz, ^ erős le^szek,
    ^Kertedként ^áp{\_{o}}^lom a ^lelke^met.
  \endverse

  \refrain

  \ifchorded
    \beginverse*
      {\nolyrics
        Utójáték: \[G] \[Am] \[Em] \[C] \[G-Em] \[Am-D] \[G] \[D]
        \hspace{1.49cm} \[G] \[Am] \[Em] \[C] \[G-Em] \[Am-D] \[G]
      }
    \endverse
  \fi

\endsong

\beginsong{Menjetek be kapuin}
[
  li={Kék könyv / 146.}
]

  \beginverse*
    /: \[Am]Menjetek be kapuin \[G]hálaa\[Am]dással,
    Tornáca\[G]in át dicséretek\[Am]kel! :/ \rep{2}
  \endverse

  \beginverse*
    /: Mert jó az Úr, \[G]örökké való ke\[Am]gyelme és
    Hűsége \[G]megmarad mindörök\[Am]re! :/ \rep{2}
  \endverse

\endsong

\beginsong{Nagy vagy Urunk}
[
  by={Chris Tomlin / Jesse Reeves / Ed Cash | Magyar Szöveg: Marika Payne}
]

  \beginverse
    Ő \[A]tündöklő Király, \[F#m]fénye ragyog ránk,
    Hadd örvendjen a \[D2]Föld, örvendjen a Föld.
    Ő \[A]fényben lakozik, s a \[F#m]sötét eltűnik,
    Hangjára megre\[D2]meg, ha szól az megremeg.
  \endverse

  \beginchorus
    Mily’ \[A]nagy vagy Urunk, énekeljük:
    \[F#m]Nagy vagy Urunk, mind meglátjuk,
    \[Dmaj7]Nagy, mily’ n\[E]agy vagy Ur\[A]unk.
  \endchorus

  \beginverse
    ^Örökké létező, ^kezében az idő
    A kezdet és a ^vég, a kezdet és a vég.
    Az ^egyetlen Isten, ki ^Atya, Fiú, Lélek,
    A megölt Bá^rány, a Győztess Oroszlán.
  \endverse

  \beginverse*
    Refrén
  \endverse

  \beginverse*
    /: \[A]Méltó nagy Neved,
    \[F#m]Minden név felett,
    A \[Dmaj7]szívünk áld,
    Mily’ n\[E]agy vagy Ur\[A]unk. :/
  \endverse

  \beginverse*
    Refrén
  \endverse

\endsong

\beginsong{Szentlélek}
[
  li={Kék könyv / 190. (le 4)}
]

  \beginchorus
    Szent\[D2]lé\[D]lek, úgy \[D2]kérünk, szállj le r\[D]ánk,
    Töltsd el a \[G]szívünk, \[Em] éle\[A]tünk,
    Hogy \[D2]bé\[D]ke és \[D2]áldás szálljon \[D]ránk,
    Küldd el a \[G]Lelked, úgy \[A]kérünk, Isten\[D]ünk.
  \endchorus

  \beginverse\memorize
    \[D]Kegyelmeddel \[A]táplálsz, Isten\[Hm]ünk, \[G]
    Hozzád \[Em]emeljük most mindnyájan a \[A]szívünk,
    \[D]Reménységünk \[A]Belőled fa\[Hm]kad, \[G]
    Hogyha \[Em]szívünk-lelkünk mindentől sza\[A]bad.
  \endverse

  \refrain

  \beginverse
    ^Ajándékul ^Lelked küldted e^l, ^
    Hogy ne ^önmagunknak, hanem annak ^éljünk,
    ^Aki értünk ^életét ad^ta. ^
    Mi is ^átadjuk most Néked élet^ünk.
  \endverse

  \refrain{\\Küldd el a \[G]Lelked, úgy \[A]kérünk, Isten\[D]ünk.}

\endsong

\beginsong{Szívem telve van Veled}
[
  %by={Szerző},
  %sr={Referencia},
  %cr={\copyright~2015 XYZ.},
  %li={Used with permission.},
  %index={extra index entry for a line of lyrics},
  %ititle={an extra index entry for a hidden title}
]

  \ifchorded
    \beginverse*
      {\nolyrics Előjáték: \[E] \[G#m\⁷] \[A/H]}
    \endverse
  \fi

  \beginverse*
    \[E] A szívem \[G#m]telve van Ve\[C#m]led, \[C#m\⁹] drága J\[A]ézus, \[E] hű Megv\[H]áltóm.
    \[E] Hálát \[G#m]érzek mindaz\[C#m]ért, \[C#m\⁹] mit értem t\[A]ettél, \[E] drága J\[H]ézus.
    \[F#m] Hogy elhív\[C#m]tál és \[H]befogad\[C#m]tál,
    \[F#m] Nincsen \[C#m]más ottho\[H]nom.
    Karod \[E]óv és \[G#m]átöl\[A]el,
    Nálad \[E]van \[G#m]lakhely\[A]em.
    Jöjj közel \[C#m]hát és \[G#m]ölelj most \[A]át,
    Szüksé\[H]gem \[A]van {\[E]R}{\[G\shrp m C\shrp m]á}d!
    Szüksé\[H]gem van {\[E]R}{\[G\shrp m C\shrp m]á}d!
    Szüksé\[H]gem van \[E]Rád!
  \endverse

\endsong

\beginsong{Utazós dal}
[
  by={Szabó András}
]

  % https://youtu.be/eAXXsMdLEdA
  % https://youtu.be/E6oXu26MOAE

  \transpose{2}

  \beginverse
    Ha a \[C]Lelked viszi \[F]léptem rögös \[Am]ösvényei\[G]den,
    Jöhet \[C]tűz, ár, jöhet \[F]orkán, az u\[Am]tam nem \[G]vész \[C]el.
  \endverse

  \beginverse
    Magas ^erdő, színes ^égbolt hajol ^ösvényem^re,
    Ez a ^szépség új ^háladalt ^ír szí^vem^be.
  \endverse

  \beginchorus
    Ez az \[C/E]út, mit nekem \[F]adtál gyönyö\[Am]rű s néha \[G]fáj,
    De e\[C]lőttem a ke\[F]reszted és a \[Am]csilla\[G]god \[C]már.
  \endchorus

  \beginverse
    Ha az ^éjjel hideg ^árnya lepi ^ösvényem ^el
    Csak a ^Hajnalcsil^lagfény ragyog ^és ve^zér^el.
  \endverse

  \beginverse
    Szaka^dékból helyes ^útra aki ^elvezet^tél,
    A ve^zérem, úti^társam, úti^célom ^let^tél.
  \endverse

  \beginverse*
    \textbf{\textit{Refrén}}
  \endverse

  \beginverse*
    /: Veled \[Am]lendüljön a \[F]léptünk, Neked \[C]zendüljön a \[G]szívünk,
    Veled \[Am]lendüljön a \[F]léptünk, ó \[C]hű Vezé\[G]rünk! :/ \rep{2}
  \endverse

  \beginverse*
    \textbf{\textit{Refrén}}
  \endverse

  \beginchorus
    Ez az \[C/E]út, mit nekem \[F]adtál gyönyö\[Am]rű s néha \[G]fáj,
    De e\[C]lőttem új \[F]égbolt és \[Am]új föld \[G]mi \[C]vár.
  \endchorus

\endsong



    % \setcounter{songnum}{888}
% \beginsong{Zsoltár - 2021.02.06.}[sr={Zsolt 146,1-2.3-4.5-6}]
%
%   \beginchorus
%     \[G]Dicsérjétek az Urat, \[C]* aki a megtört szívűeket meggyógyít\[Em]ja.
%   \endchorus
%
%   \beginverse*
%     Dicsérjétek az Urat, mert jó dolog őt áldani, * Istenünknek kedves a dicséret.
%     Jeruzsálemet az Úr újjáépíti, * Izrael szétszórt fiait egybegyűjti.
%   \endverse
%
%   \beginchorus
%     \[G]Dicsérjétek az Urat, \[C]* aki a megtört szívűeket meggyógyít\[Em]ja.
%   \endchorus
%
%   \beginverse*
%     Aki a megtört szívűeket meggyógyítja, * és sebeiket bekötözi.
%     A csillagokat számon tartja, * és nevén szólítja valamennyit.
%   \endverse
%
%   \beginchorus
%     \[G]Dicsérjétek az Urat, \[C]* aki a megtört szívűeket meggyógyít\[Em]ja.
%   \endchorus
%
%   \beginverse*
%     Nagy a mi Urunk, nagy az ő hatalma, * bölcsességének nincs határa.
%     A szelídeket magához emeli, * de porig alázza a bűnösöket.
%   \endverse
%
%   \beginchorus
%     \[G]Dicsérjétek az Urat, \[C]* aki a megtört szívűeket meggyógyít\[Em]ja.
%   \endchorus

  % \ifchorded
  %   \beginverse*
  %     {\nolyrics
  %       \[G] \[A] \[G] \[G] \[F#] \[D] \[E]
  %     }
  %   \endverse
  % \fi
% \endsong

% \setcounter{songnum}{999}
% \beginsong{Igevers - 2021.02.06.}[sr={Mt 8,17}]
%
%   \beginverse*
%     \[F#]Krisztus gyengeségünket magára vette, \[Hm]* és betegségeinket ő hor\[C#]dozta.
%   \endverse
%
% \endsong

% ------------------------------------------------------------------------------

% \setcounter{songnum}{999}
% \beginsong{Alleluja}[]
%
%   \beginchorus
%     Alleluja, alleluja…
%   \endchorus
%
%   \beginverse*
%     Készítsétek elő az Úr útját!
%     Egyengessétek ösvényét,
%     És minden ember meglátja az Üdvözítőt, akit elküld az Isten.
%   \endverse
%
%   \beginchorus
%     Alleluja, alleluja…
%   \endchorus
%
% \endsong

% ------------------------------------------------------------------------------

% \setcounter{songnum}{999}
% \beginsong{Alleluja}
% [
%   by={Gocam ??? Varga Attila},
%   li={Kék könyv / 230. [-2]}
% ]

%   \beginchorus
%     /: \[Hm]Alleluja, \[A]alleluja, \[G]allelu\[F#]ja! :/ \rep{2}
%   \endchorus

%   \beginverse
%     \[D]Ó, Urunk, mutassad \[A]meg nekünk, \[Hm]irgalmas szíve\[F#]det,
%     \[D]És az üdvösséget \[A]add nekünk, \[Hm]add meg \[G]nékünk, Iste\[F#]nünk!
%   \endverse

%   \beginchorus
%     /: \[Hm]Alleluja, \[A]alleluja, \[G]allelu\[F#]ja! :/ \rep{2}
%   \endchorus

% \endsong

% ------------------------------------------------------------------------------

% \setcounter{songnum}{888}
% \beginsong{Zsoltár}
%
%   \beginchorus
%     \[G]Uram, Iste\[Em]nem, * örök \[G]életet adó igé\[D]id van\[Em]nak.
%   \endchorus
%
%   \beginverse
%     \[G]Tökéletes az Úr t\[Em]örvénye, * \[G]megújítja \[D]a lelk\[Em]et.
%     Az Úr tanúságában bízhatunk, * bölcsességet ad a kicsinyeknek.
%   \endverse
%
%   \beginchorus
%     \[G]Uram, Iste\[Em]nem, * örök \[G]életet adó igé\[D]id van\[Em]nak.
%   \endchorus
%
%   \beginverse
%     Az ^Úr végzése ^igaz, * ^megvidámítja ^a szív^et.
%     Az Úr parancsa világos, * fényt gyújt a szemnek.
%   \endverse
%
%   \beginchorus
%     \[G]Uram, Iste\[Em]nem, * örök \[G]életet adó igé\[D]id van\[Em]nak.
%   \endchorus
%
%   \beginverse
%     Az ^Úr félelme ^tiszta: * ^örökké ^megmar^ad.
%     Az Úr ítéletei igazak, * és mind egyaránt jogosak.
%   \endverse
%
%   \beginchorus
%     \[G]Uram, Iste\[Em]nem, * örök \[G]életet adó igé\[D]id van\[Em]nak.
%   \endchorus
%
%   \beginverse
%     ^Értékesebb az ^aranynál, * ^és a legdrágább drá^gakőn^él.
%     Édesebb a méznél, * a csorduló lépes méznél.
%   \endverse
%
%   \beginchorus
%     \[G]Uram, Iste\[Em]nem, * örök \[G]életet adó igé\[D]id van\[Em]nak.
%   \endchorus
%
% \endsong
%
% \setcounter{songnum}{999}
% \beginsong{Igevers}
%
%   \beginverse
%     \[Dm]Úgy szerette Isten a vi\[F]lágot, hogy \[A#]egyszülöttét adta \[C]érte
%     Hogy \[C]mindaz aki hisz \[F]Őbenne, \[F]el ne vesszen, hanem \[Gm]örökké \[Dm]éljen.
%   \endverse
%
% \endsong

% ------------------------------------------------------------------------------

\setcounter{songnum}{888}
\beginsong{Zsoltár - Irgalmas Istenünk}

  \versesep=12pt plus 3pt minus 8pt

  \beginchorus
    \[Dm] Irgalmas \[A]Istenünk \[Dm]jósá\[C]gát \[F]mindö\[C]rökké \[Dm]é\[A]nek\[Dm]lem!
  \endchorus

  \beginverse
    Mondja Izrael háza, hogy jó az Isten, * mivel irgalma örökké megmarad.
    Mondja Áron papi háza is, * mivel irgalma örökké megmarad.
    Mondják, akik félik az Urat, * mivel irgalma örökké megmarad.
  \endverse

  \beginchorus
    \[Dm] Irgalmas \[A]Istenünk \[Dm]jósá\[C]gát \[F]mindö\[C]rökké \[Dm]é\[A]nek\[Dm]lem!
  \endchorus

  \beginverse
    Az Úr jobb keze erősnek bizonyult, * az Úrnak jobbja fölemelt engem.
    Nem halok meg, hanem élek, * és hirdetem az Úr nagy tetteit.
    Megfenyített az Úr és megpróbált, * de nem adott halálra engem.
  \endverse

  \beginchorus
    \[Dm] Irgalmas \[A]Istenünk \[Dm]jósá\[C]gát \[F]mindö\[C]rökké \[Dm]é\[A]nek\[Dm]lem!
  \endchorus

  \beginverse
    A kő, amelyet az építők félredobtak, * íme, az vált szegletkővé.
    Mindezt az Úr vitte végbe, * csodálatra méltó a mi szemünkben.
    Ezt a napot az Úristen adta, * örvendjünk és vigadjunk rajta.
  \endverse

  \beginchorus
    \[Dm] Irgalmas \[A]Istenünk \[Dm]jósá\[C]gát \[F]mindö\[C]rökké \[Dm]é\[A]nek\[Dm]lem!
  \endchorus

\endsong

\setcounter{songnum}{999}
\beginsong{Igevers - Alleluja}
  \transpose{3}
  \prefersharps
  \beginchorus
    Alle\[Hm]luja, alle\[D]luja, alle\[G]l\[A]{\_{u}}\[D]ja. Alle\[Hm]luja, alle\[D]luja, alle\[G]l\[Em F#]{\_{u}}\[Hm]ja!
  \endchorus

  \beginverse*
    % \hspace{-1cm}Most már hiszel, Tamás, mert láttál engem. Boldogok, akik nem láttak, és mégis hittek.
    Most már hiszel, Tamás, mert láttál engem.
    Boldogok, akik nem láttak, és mégis hittek.
  \endverse

  \beginchorus
    Alle\[Hm]luja, alle\[D]luja, alle\[G]l\[A]{\_{u}}\[D]ja.  Alle\[Hm]luja, alle\[D]luja, alle\[G]l\[Em F#]{\_{u}}\[Hm]ja!
  \endchorus

\endsong


    % \renewcommand{\thesongnum}{T\arabic{songnum}}
    % \setlength{\songnumwidth}{1.1cm}
    \beginsong{Ahol szeretet - !!}
[
  by={Taizé / Jacques Berthier}
]

  \beginverse*
    \[F]Ahol \[C]szere\[Dm]tet és \[A#]j\[D]ó\[G]ság, \[F]ahol \[C]szere\[Dm]tet, \[Gm]ott van \[C]Iste\[F]nünk.
  \endverse

\endsong

\beginsong{Áldott légy, Uram}
[
  by={Jacques Berthier}
]

  \beginverse*
    \[Dm]Áldott \[G]légy, U\[Dm]ram, szent \[A#]neved \[C]áldja \[F]lel\[A]kem!
    \[Dm]Áldott \[G]légy, U\[Dm]ram, mert \[A#]megvál\[C]tottál \[Dm]már.
  \endverse

\endsong

\beginsong{Bizakodjatok, jó az Úr - !!}
[
  by={Taizé / Jacques Berthier}
]

  \beginverse*
    \[D]Bizakodjatok, \[Hm]jó az Úr, \[D]jósága \[A]éltet,
    \[Em]Bizakodjatok, \[C]jó az Úr, \[Em]Allel\[A]u\[D]ja!
  \endverse

\endsong

\beginsong{Csak vándorolunk}
[
  by={Jacques Berthier}
]

  \beginverse*
    Csak \[Dm]vándorolunk az \[A#]éjben, mert \[C\⁶]forrás vi\[Gm/A#]zére \[A\⁴]vá\[A]gyunk.
    \[Dm]Szomjunk a \[C]fény a sö\[F]{t\_{é}}t\[A]ben, szomjunk a \[A#]fény a sö\[A]tétben.
  \endverse

\endsong

\beginsong{Gyújts éjszakánkba fényt}
[
  by={Jacques Berthier}
]

  \beginverse*
    \[H]Gyújts éjszakánkba \[Em]fényt, hadd égjen a
    Soha ki nem al\[D]vó \[G]tűz, a ki nem \[C]al\[G]vó \[D]tűz!
    Gyújts \[G]éjszakánkba \[C]fényt, hadd égjen a
    \[H]Soha ki nem \[Em]al\[Am6]vó \[H]tűz, a ki nem \[Em]al\[Am6]vó \[H]tűz!
  \endverse

\endsong

\beginsong{Irgalmas Istenünk - !!}
[
  by={Taizé / Jacques Berthier}
]

  \beginverse*
    \[Dm] Irgalmas \[A]Istenünk \[D]jósá\[C]gát \[F]mindö\[C]rökre \[Dm]é\[A]nek\[Dm]lem!
  \endverse

\endsong

\input{songs-taize/jezus-eletem.tex}
\beginsong{Jézus, majd gondolj rám - !!}
[
  by={Taizé / Jacques Berthier}
]

  \beginverse*
    \[D]Jézus, majd \[Em7]gondolj rám, \[A]ha a Te országod \[D]eljön.
    \[Hm]Jézus, majd \[Em]gondolj rám, \[A]ha a Te országod \[D]eljön.
  \endverse

\endsong

\beginsong{Jó az Úrban bizakodni}
[
  by={Taizé / Jacques Berthier}
]

  \beginverse*
    \[Dm]Jó az Úrban \[A]bizakodni, {\[Dm]j}\[C]{\_{ó}} az \[F]Úr.
    \[Gm]Re\[C]mélj, és \[F]b{\_{í}}zz \[C]Ben\[Dm]ne, {\[A#]j}\[C]{\_{ó}} az \[Dm]Úr!
  \endverse

\endsong

\beginsong{Ne félj, ne aggódj}
[
  by={Taizé / Jacques Berthier}
]

  \beginverse*
    \[Am]Ne félj, ne \[Dm7]aggódj, \[G]ne sírj, ne \[Cmaj7]bánkódj:
    \[F]Ha tiéd \[Dm6]Isten, \[E]tiéd már \[Am]minden.
    \[Am]Ne félj, ne \[Dm7]aggódj, \[G]ne sírj, ne \[Cmaj7]bánkódj:
    \[F]Elég \[Dm6]Ő \[E]né\[Am]ked.
  \endverse

\endsong

\beginsong{Az Úrra vár a szívünk}[]

  \beginverse*
    Az \[Dm]Úrra \[C]vár a szí\[F]vünk,
    Ő\[Gm]benne \[Dm]minden \[A]örö\[Dm]münk.
  \endverse

  \ifchorded
    \beginverse*
      {\nolyrics ( Közjáték: \[Dm] \[A#] \[C]
      \[F] \[Dm] \[Em] \[A] )}
    \endverse
  \fi

\endsong

\beginsong{Várj és ne félj}
[
  by={Jacques Berthier}
]

  \beginverse*
    \[Em]Várj és ne \[C]félj, az \[Am\⁶]Úr jön \[H\⁷]már!
    \[Em]Várj \[D]és ne \[G]félj, hű \[Am\⁷]szív\[H]vel \[Em]várj!
  \endverse

\endsong



    % \renewcommand{\thesongnum}{K\arabic{songnum}}
    \renewcommand{\thesongnum}{K\arabic{songnum}}

\beginsong{Hála Néked, Istenünk - !!}[]

  \beginverse*\memorize
    \[G]Hála Néked, \[C]Istenünk, \[G]hála néked \[D]………… -ért!
    \[G]Hála Néked, \[C]Istenünk, mert \[G]ő fon\[D]tos ne\[G]künk! \[D] \[G]
  \endverse

  \beginverse*
    ^Alleluja, ^áldd Uram, ^alleluja, ^áldd Uram,
    ^Alleluja, ^áldd Uram, mert ^ő fon^tos ne^künk! ^ ^
  \endverse

  \beginverse*
    Pamparararam \[D]pam \[G]pam paramparampampam. \[G]
  \endverse

\endsong

\beginsong{Köszönjük Néked, Urunk - !!}[]

  \beginverse*\memorize
    \[G]Köszönjük neked, \[C]Urunk, ………..-t, \[G] köszönjük neked \[D]örökké,
    \[G]Ó, Urunk, mi \[C]nem hagyjuk el Őt, és \[G]Veled \[D]dalol\[G]juk: \echo{Együtt, mindenki!}
  \endverse

  \beginverse*
    ^Yep ye-e-e-^ep ye-e-e-^ep ye-e-e-^ep oh oh oh!
    ^Yep ye-e-e-^ep ye-e-e ^yep oh ^yep oh ^yep \[D]oh \[G]yep!
  \endverse

  \beginverse*
    Pamparararam \[D]pam \[G]pam paramparampampam. \[G]
  \endverse

\endsong

\beginsong{Jó Atyánk, köszönjük Néked}[]

  \ifchorded
    \beginverse*
      {\nolyrics Előjáték: /: \[D] \[D4] \[D] \[D2] :/ \rep{2}}
    \endverse
  \fi

  \beginverse*\memorize
    \[D]Jó Atyánk, köszönjük Néked \[G]\_\_\_\_\_-t,
    Szívünkbe zártuk, \[D]hála \[D4]érte! \[D] \[D2]
  \endverse

  \beginverse*
    ^Áldj meg minket, Krisztusunk, a  ^Lelked legyen
    Közöttünk míg ^együtt ^vagyunk! ^ ^
  \endverse

  \ifchorded
    \beginverse*
      {\nolyrics \[H7]}
    \endverse
  \fi

  \beginverse*\memorize
    \[E]Ó, Urunk, mi \[A]egyet aka\[E]runk, kik
    Ebben a dalban \[H7]együtt vagyunk:
  \endverse

  \beginverse*
    ^Szereteted fénye, á^ldása érje ^\_\_\_\_\_ -t!
    Legyen a ^Lelked véle! \[E]Alle\[A]lu\[E]ja!
  \endverse

\endsong



    % \renewcommand{\thesongnum}{M\arabic{songnum}}
    % \setlength{\songnumwidth}{1.25cm}
    \renewcommand{\thesongnum}{M\arabic{songnum}}
\setlength{\songnumwidth}{1.25cm}

\setcounter{songnum}{10}
\beginsong{Uram, irgalmazz I.}
[
  by={Sillye Jenő},
  li={Kék könyv / 159.}
]

  \ifchorded
    \beginverse*
      {\nolyrics Előjáték:
        /: \[Dm] \[Gm] \[C7] \[F] :/ \rep{2}
        \[Dm] \[Gm] \[A7] \[Dm]
      }
    \endverse
  \fi

  \beginverse*
    \[Dm] Uram irgal\[Gm]mazz! \[C7] Uram irgal\[F]mazz!
    \[Dm] Krisztus kegyel\[Gm]mezz! \[C7] Krisztus kegyel\[F]mezz!
    \[Dm] Uram irgal\[Gm]mazz! \[A7] Uram irgal\[Dm]mazz!
  \endverse

  \ifchorded
    \beginverse*
      {\nolyrics Utójáték:
        \[Dm] \[Gm] \[C7] \[F]
        \[Dm] \[Gm] \[A7] \[Dm]
      }
    \endverse
  \fi

\endsong

\beginsong{Uram, irgalmazz II.}
[
  index={Jézus Krisztus, Téged elküldött az Atya},
  li={Kék könyv / 129.B}
]

  \ifchorded
    \beginverse*
      {\nolyrics Előjáték: \[A\⁶] \[G\⁶] \[A] \[G-D]}
    \endverse
  \fi

  \beginverse\memorize
    \[A\⁶]Jézus Krisztus, Téged \[G\⁶]elküldött az Atya,
    Hogy \[A]gyógyítsd a megtört \[D]szívűeket. \[D\⁷]
    /: Uram, \[G]irgalmazz nekünk!
    Uram, \[D]irgalmazz ne\[Hm]künk!
    Uram, \[Em]irgal\[A]mazz nek\[G]ünk! \[D] :/ \rep{2}
  \endverse

  \beginverse
    ^Jézus Krisztus, Te ^eljöttél hozzánk,
    Hogy ^hívjad a bűnös ^embereket. ^
    /: Krisztus, ^kegyelmezz nekünk!
    Krisztus, ^kegyelmezz ne^künk!
    Krisztus, ^kegyel^mezz nek^ünk! ^ :/ \rep{2}
  \endverse

  \beginverse
    ^Jézus Krisztus, Te az ^Atya jobbján ülsz,
    Hogy ^értünk örökre ^közbenjárj. ^
    /: Uram, ^irgalmazz nekünk!
    Uram, ^irgalmazz ne^künk!
    Uram, ^irgal^mazz nek^ünk! ^ :/ \rep{2}
  \endverse

\endsong


\setcounter{songnum}{20}
\beginsong{Dicsőség I.}
[
  by={Sillye Jenő},
  li={Kék könyv / 159.}
]

  \beginverse*
    \[F] \[F]Dicsőség a \[Gm]magasságban \[C\⁷] Isten\[F]nek,
    \[F]És a Földön \[Gm]békesség a \[C\⁷]jószándékú \[Dm]embernek! \[Dm]
    \vspace{0.2cm}
    \[A#]Dicsőítünk \[C\⁷]Téged, \[Gm] áldunk \[Dm]Téged,
    \[A#] Imádunk \[C\⁷]Téged, \[Gm]magasztalunk \[Dm]Téged,
    \vspace{0.2cm}
    \[C\⁷]Hálát adunk \[F]Neked \[Gm] nagy dicső\[F]ségedért,
    \[F] Urunk és \[Gm]Istenünk, \[C\⁷] mennyei Ki\[F]rály,
    \[Gm]Mindenható \[C\⁷]Atyais\[F]ten! \[Gm] \[C\⁷] \[F]
    \vspace{0.2cm}
    \[A#]Urunk, Jézus \[C\⁷]Krisztus, \[Gm]egyszülött Fi\[Dm]ú,
    \[A#] Urunk és \[C\⁷]Istenünk, \[Gm] Isten \[Dm]Báránya,
    \vspace{0.2cm}
    Az \[C\⁷]Atyának Fi\[F]a, Te elveszed a \[E]világ bűne\[A]it, \[A] \[Dm] irgalmazz \[Gm]nékünk; \[Gm]
    Te \[F]elveszed a \[E]világ bűne\[A]it, \[A] \[Dm] hallgasd meg könyörgésün\[Gm]ket.
    Te az \[F]Atya jobbján \[E]ülsz, \[E] irgalmazz \[A]nékünk!
    Mert \[D]egyedül Te vagy a Szent, Te vagy az \[A]Úr,
    Te vagy az \[F#m]egyetlen Föl\[E]ség, Jézus \[A]Krisztus,
    A \[F#]Szentlélekkel \[Hm]együtt, az \[E]Atyaisten dicsőségé\[A]ben. \[G]
    \[A]{\_{A}}\[D]men.
  \endverse

\endsong


\setcounter{songnum}{30}
\beginsong{Szent vagy I.}
[
  by={Illésy István},
  li={Kék könyv / 192.}
]

  \ifchorded
    \beginverse*
      {\nolyrics Előjáték: \[F] \[Em] \[Am] \[Dm] \[G] \[G\⁷] \[C]}
    \endverse
  \fi

  \beginverse\memorize
    Szent vagy, \[C]szent vagy, szent vagy, \[G\⁷]sz{\_{e}}nt vagy,
    Minden\[F]ségnek Ura, \[G]Istene, \[C]
    \[C\⁷] Dicső\[F]séged betölti a menny\[Em]et és a Föld\[Am]et,
    Hozsan\[Dm]na \[G] a magas\[G\⁷]ság\[C]ban!
  \endverse

  \beginverse
    Szent vagy, ^szent vagy, szent vagy, ^szent vagy,
    Minden^ségnek Ura, ^Istene, ^
    ^ \_{Ó} ^áldott, ki az Úr nevében ^eljön mihoz^zánk,
    Hozsan^na ^ a magas^ság^ban!
  \endverse

  \ifchorded
    \beginverse*
      {\nolyrics Utójáték: \[F] \[Em] \[Am] \[Dm] \[G] \[G\⁷] \[C]}
    \endverse
  \fi

\endsong


\setcounter{songnum}{40}
\beginsong{Isten báránya I.}
[
  by={Ferenczy Rudolf},
  li={Kék könyv / 299.}
]

  \ifchorded
    \beginverse*
      {\nolyrics Előjáték: \[Em] \[Am] \[Ebdim] \[Em]}
    \endverse
  \fi

  \beginverse
    \[Em]Bűneinket, égi Bárány, \[Am]szent vérednek \[D7]drága árán
    \[G]mind elveszed, \[Am]irgal\[H7]mazz ne\[Em]kem!
  \endverse

  \beginverse
    \[Em]Bűneinket, égi Bárány, \[Am]szent vérednek \[D7]drága árán
    \[G]mind elveszed, \[Am]irgal\[H7]mazz ne\[Em]künk!
  \endverse

  \beginverse
    \[Em]Bűneinket, égi Bárány, \[Am]szent vérednek \[D7]drága árán
    \[G]mind elveszed, \[Am]adj bé\[H7]két ne\[Em]künk!
  \endverse

  \ifchorded
    \beginverse*
      {\nolyrics Utójáték: \[Em] \[Am] \[Ebdim] \[Em]}
    \endverse
  \fi

\endsong



    \renewcommand{\thesongnum}{???}
    \beginsong{Alleluja}\endsong

  \end{songs}
  \newpage
  \thispagestyle{empty}
  % To be continued...
  \vspace*{\fill}
  % \vfill
  Budaörs, 2022. június 12.
\end{document}
